\section{CycleGAN: Ergebnisse und objekte Bewertung}

Im vorliegenden Abschnitt liegt der Fokus auf der objektiven Evaluation der Leistung des CycleGAN-Algorithmus. Hierzu werden verschiedene Metriken und Kriterien herangezogen, darunter die Verluste der Generatoren und Diskriminatoren unter Variation der Hyperparameter. Zudem erfolgt eine eingehende Analyse der Genauigkeit des Diskriminators sowie der SSIM-Score. Diese quantitativen Maße dienen dazu, die Qualität der generierten Ergebnisse präzise zu erfassen und die Lernfähigkeit der Modelle umfassend zu beurteilen.

Die für die Evaluation genutzten Testdatensätze stammen aus dem Berkeley-Repository. Besondere Beachtung finden dabei die Datensätze 'horse2zebra' und 'maps', bei denen Transformationen von Pferden zu Zebras und vice versa sowie von Kartenansichten zu Satellitenbildern und umgekehrt durchgeführt wurden.