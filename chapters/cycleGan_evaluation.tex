Die Datensätze $X$ (Karten) und $Y$ (Satellitenbilder) werden durch die Generatoren $G: X\rightarrow Y$ und $F: Y\rightarrow X$ sowie die Diskriminatoren $D_X$ und $D_Y$ repräsentiert, die jeweils darauf abzielen, Bilder in den Domänen $X$ und $Y$ zu unterscheiden. Die Hyperparameter basieren zunächst auf den in dem Originalpaper von Zhu ausgewählten Parametern und wurden anschließend durch explorative Anpassungen verfeinert.
\\\newline
Ein herausforderndes Problem, dem sich die Generatoren gegenübersehen, besteht in der Reproduktion feiner und detaillierter Strukturen. Insbesondere fällt auf, dass die Generatoren besser darin sind, Straßenlinien zu erzeugen, was auf die Dominanz von Straßen- und Häusermotiven in den Trainingsdaten zurückzuführen ist.
Die Vielfalt in den Bildern, wie Flüsse und Wälder, stellt hingegen eine größere Herausforderung dar. Die Generatoren haben Schwierigkeiten, die Struktur und Farbkodierung dieser Bereiche präzise zu erfassen. Häufig resultiert dies in der Generierung von Flächen mit ähnlicher Farbe wie Straßen, und es werden Versuche unternommen, basierend auf diesen großen Freiflächen kleine Straßen zu generieren.
\\\newline
Die Wahl der Lernrate erwies sich als kritisch, wobei die besten Ergebnisse bei einer Lernrate von $0,0002$ erzielt wurden. Diese Einstellung führte zu den höchsten SSIM-Werten und Diskriminator-Testgenauigkeiten. Insbesondere erzielte der Generator $F$ einen SSIM-Score von $0.592$, während der Generator $G$ einen Wert von $0.198$ erreichte. Die Diskriminatoren zeigten mit $50.1\%$ für $D_Y$ und $48,5\%$ für $D_X$ nahezu optimale Genauigkeiten. Trotz der zunächst hoch erscheinenden visuellen Ähnlichkeit zwischen den generierten Satellitenbildern zeigte eine genauere Analyse, dass kleinere Strukturen im generierten Bild fehlten. Dies spiegelte sich in den vergleichsweise niedrigen SSIM-Wert wider.



\begin{figure}
  \begin{subfigure}[t]{.3\textwidth}
    \centering
    \caption{Input}
    \includegraphics[width=\linewidth]{example-image-a}
  \end{subfigure}
  \hfill
  \begin{subfigure}[t]{.3\textwidth}
    \centering
    \caption{Output}
    \includegraphics[width=\linewidth]{example-image-b}
  \end{subfigure}
  \hfill
  \begin{subfigure}[t]{.3\textwidth}
    \centering
    \caption{Differenzbild}
    \includegraphics[width=\linewidth]{example-image-b}
  \end{subfigure}

  \medskip

  \begin{subfigure}[t]{.3\textwidth}
    \centering
    \includegraphics[width=\linewidth]{example-image-a}
  \end{subfigure}
  \hfill
  \begin{subfigure}[t]{.3\textwidth}
    \centering
    \includegraphics[width=\linewidth]{example-image-b}
  \end{subfigure}
  \hfill
  \begin{subfigure}[t]{.3\textwidth}
    \centering
    \includegraphics[width=\linewidth]{example-image-b}
  \end{subfigure}
  \caption{Testergebnisse nach dem Training mit $\lambda=120$, Lernrate $=0.0002$ und $100$ Epochen}
  \label{evaluation:cycleGan_Testergebnisse}
\end{figure}

\begin{figure}
  \begin{subfigure}[t]{.24\textwidth}
    \centering
    \includegraphics[width=\linewidth]{example-image-a}
    \caption{(a) Generator-Verlust}
  \end{subfigure}
  \begin{subfigure}[t]{.24\textwidth}
    \centering
    \includegraphics[width=\linewidth]{example-image-b}
    \caption{(b) Diskriminator-Metriken}
  \end{subfigure}
  \begin{subfigure}[t]{.24\textwidth}
    \centering
    \includegraphics[width=\linewidth]{example-image-b}
    \caption{(c) SSIM-Score}
  \end{subfigure}
  \begin{subfigure}[t]{.24\textwidth}
    \centering
    \includegraphics[width=\linewidth]{example-image-b}
    \caption{(d) SSIM-Score Zyklus}
  \end{subfigure}

  \caption{Metriken nach dem Training mit $\lambda=120$, Lernrate $=0.0002$ und $100$ Epochen}
  \label{evaluation:cycleGan_Metriken}
\end{figure}


Ebenso zeigten die Bilder bei einem Gewichtungsfaktor von $\lambda = 120$ die besten Resultate, sowohl basierend auf visueller Beurteilung als auch auf dem SSIM-Score (\ref{evaluation:cycleGan_Testergebnisse}). Dieser spezifische Wert für $\lambda$ führte zu generierten Bildern, deren Signal-Rausch-Verhältnis (SNR) entweder besser oder nahezu dem des Originalbildes entspricht. Der SNR-Wert gibt dabei das Verhältnis zwischen dem Signal, repräsentiert durch die relevanten Bildinformationen, und dem Rauschen, repräsentiert durch unerwünschte Störungen, an. In diesem Zusammenhang zeigt ein höherer SNR-Wert eine bessere Qualität und Klarheit der generierten Bilder im Vergleich zu Rauschen.
\\\newline
Trotz dieser Erfolge bleibt das Training instabil, und es besteht die Möglichkeit eines Modekollapses. Die Generatoren zeigen Anzeichen eines exponentiellen Verfalls in ihren Verlustkurven, während der SSIM-Score nur langsam ansteigt. Dies deutet darauf hin, dass die Generatoren zwar ihre Verluste minimieren, jedoch möglicherweise nicht die gewünschte visuelle Qualität erreichen (\ref{evaluation:cycleGan_Metriken}).

\begin{figure}
  \begin{subfigure}[t]{.2\textwidth}
    \caption{Input}
    \centering
    \includegraphics[width=\linewidth]{example-image-a}
  \end{subfigure}
  \begin{subfigure}[t]{.2\textwidth}
    \caption{Output}
    \centering
    \includegraphics[width=\linewidth]{example-image-b}
  \end{subfigure}
  \hfill
  \begin{subfigure}[t]{.2\textwidth}
    \caption{Input}
    \centering
    \includegraphics[width=\linewidth]{example-image-a}
  \end{subfigure}
  \begin{subfigure}[t]{.2\textwidth}
    \caption{Output}
    \centering
    \includegraphics[width=\linewidth]{example-image-b}
  \end{subfigure}

  \medskip

  \begin{subfigure}[t]{.2\textwidth}
    \centering
    \includegraphics[width=\linewidth]{example-image-a}
  \end{subfigure}
  \begin{subfigure}[t]{.2\textwidth}
    \centering
    \includegraphics[width=\linewidth]{example-image-b}
  \end{subfigure}
  \hfill
  \begin{subfigure}[t]{.2\textwidth}
    \centering
    \includegraphics[width=\linewidth]{example-image-a}
  \end{subfigure}
  \begin{subfigure}[t]{.2\textwidth}
    \centering
    \includegraphics[width=\linewidth]{example-image-b}
  \end{subfigure}

  \medskip

  \begin{subfigure}[t]{.2\textwidth}
    \centering
    \includegraphics[width=\linewidth]{example-image-a}
  \end{subfigure}
  \begin{subfigure}[t]{.2\textwidth}
    \centering
    \includegraphics[width=\linewidth]{example-image-b}
  \end{subfigure}
  \hfill
  \begin{subfigure}[t]{.2\textwidth}
    \centering
    \includegraphics[width=\linewidth]{example-image-a}
  \end{subfigure}
  \begin{subfigure}[t]{.2\textwidth}
    \centering
    \includegraphics[width=\linewidth]{example-image-b}
  \end{subfigure}

  \medskip

  \begin{subfigure}[t]{.2\textwidth}
    \centering
    \includegraphics[width=\linewidth]{example-image-a}
  \end{subfigure}
  \begin{subfigure}[t]{.2\textwidth}
    \centering
    \includegraphics[width=\linewidth]{example-image-b}
  \end{subfigure}
  \hfill
  \begin{subfigure}[t]{.2\textwidth}
    \centering
    \includegraphics[width=\linewidth]{example-image-a}
  \end{subfigure}
  \begin{subfigure}[t]{.2\textwidth}
    \centering
    \includegraphics[width=\linewidth]{example-image-b}
  \end{subfigure}
  \caption{Beste Ergebnisse des Trainings}

\end{figure}


Die generierten Bilder weisen bereits zu diesem Zeitpunkt eine grobe visuelle Ähnlichkeit mit der Struktur des zugrunde liegenden Originalbildes auf und zeigen nur minimale Variationen. Allerdings wird durch den SSIM-Score deutlich, dass die Generatoren Schwierigkeiten haben, bedeutende Verbesserungen hinsichtlich feinerer Details und Farbübereinstimmungen zu erzielen. Diese lassen sich auch visuell in den Differenzbilder, die mit Hilfe eines Histogrammsausgleich kontrastverstärkt wurden, erkennen. Helle Bereiche in diesen Differenzbildern stellen dabei größere Unterschiede dar. Dabei sammeln sich helle Bereiche vor allem bei den kleinen pixelweisen Details an, sowie bei unterschiedlichen Farbkodierungen. Ein konkretes Beispiel ist in Abbildung \ref{evaluation:cycleGan_Testergebnisse} zu sehen.
\\
Insbesondere in den frühen Epochen sind die Generatoren nicht in der Lage, Straßenlinien zuverlässig als gerade Linien zu generieren. Diese Schwächen verbessern sich im Laufe längerer Trainingszeiten, wobei bereits nach etwa 300 Epochen relativ konsistente, gerade Linien erzeugt werden.
\\\newline
Die Analyse des SSIM-Scores im Kontext der Zyklusübersetzung offenbart eine exponentielle Zunahme. Diese Beobachtung legt nahe, dass der Zykluskonsistenz-Verlust einen erheblichen Einfluss auf das Training ausübt. Die exponentielle Steigerung des SSIM-Scores deutet darauf hin, dass die Konsistenz zwischen den Original- und zurückübersetzten Bildern im Laufe der Zeit kontinuierlich zunimmt. Dies könnte auf eine fortlaufende Verbesserung der Generatoren hinsichtlich ihrer Fähigkeit zur Zyklusübersetzung hindeuten. Insgesamt erreichen die SSIM-Scores für die Zyklusübersetzung Werte von $80\%$ für $Y$ und $90\%$ für $X$ (\ref{evaluation:cycleGan_Metriken}).

\begin{table}
\centering
\begin{tabular}{|l|l|l|l|l|l|l|l|l|l|l|l|l|l|l|}
\hline
\textbf{Lambda} &
  \textbf{Epoche} &
  \textbf{Lernrate} &
  \textbf{\begin{tabular}[c]{@{}l@{}}$D_Y$ \\ Tr-Acc\end{tabular}} &
  \textbf{\begin{tabular}[c]{@{}l@{}}$D_X$ \\ Tr-Acc\end{tabular}} &
  \textbf{\begin{tabular}[c]{@{}l@{}}$D_Y$ \\ Te-Acc\end{tabular}} &
  \textbf{\begin{tabular}[c]{@{}l@{}}$D_X$ \\ Te-Acc\end{tabular}} \\ \hline
120 & 100 & 0.000025 & 0.549 & 0.494 & 0.307 & 0.497   \\ \hline
120 & 100 & 0.00005  & 0.537 & 0.491 & 0.321 & 0.489 \\ \hline
120 & 100 & 0.0001   & 0.519 & 0.490 & 0.418 & 0.485 \\ \hline
120 & 100 & 0.0002   & 0.518 & 0.495 & 0.501 & 0.485 \\ \hline
120 & 100 & 0.0004   & 0.516 & 0.487 & 0.462 & 0.496 \\ \hline
100 & 100 & 0.0002   & 0.521 & 0.491 & 0.527 & 0.481   \\ \hline
140 & 100 & 0.0002   & 0.562 & 0.492 & 0.544 & 0.469 \\ \hline
100 & 300 & 0.0002   & 0.51  & 0.484 & 0.47  & 0.676 \\ \hline
120 & 300 & 0.0002   & 0.508 & 0.495 & 0.468 & 0.709  \\ \hline
\end{tabular}
\caption{CycleGAN Ergebnisse unter verschiedenen Hyperparameter Teil1}
\label{evaluation:cycleGan_table1}
\end{table}

\begin{table}
\centering
\begin{tabular}{|l|l|l|l|l|l|l|l|l|l|l|l|l|l|l|}
\hline
\textbf{\begin{tabular}[c]{@{}l@{}}$F$\\ Loss\end{tabular}} &
\textbf{\begin{tabular}[c]{@{}l@{}}$G$\\ Loss\end{tabular}} &
  \textbf{\begin{tabular}[c]{@{}l@{}}$X$ \\ SSIM\end{tabular}} &
  \textbf{\begin{tabular}[c]{@{}l@{}}$Y$ \\SSIM \end{tabular}} &
  \textbf{\begin{tabular}[c]{@{}l@{}}$X$\\Z-SSIM\end{tabular}} &
  \textbf{\begin{tabular}[c]{@{}l@{}}$Y$ \\ Z-SSIM\end{tabular}} &
  \textbf{\begin{tabular}[c]{@{}l@{}}$X$\\SNR \end{tabular}} &
  \textbf{\begin{tabular}[c]{@{}l@{}}$Y$\\SNR\end{tabular}} \\ \hline
53.113 & 22.556 & 0.545 & 0.191 & 0.895 & 0.831 & -2.991 & 0.228  \\ \hline
44.626 & 20.417 & 0.541 & 0.191 & 0.916 & 0.836 & -0.589 & 1.071  \\ \hline
44.376 & 21.943 & 0.580 & 0.194 & 0.935 & 0.842 & -4.604 & 1.367  \\ \hline
42.489 & 22.373 & 0.592 & 0.198 & 0.921 & 0.796 & -5.308 & 0.851  \\ \hline
46.972 & 28.175 & 0.585 & 0.209 & 0.856 & 0.602 &        &        \\ \hline
34.973 & 18.837 & 0.551 & 0.194 & 0.93  & 0.83  & -3.292 & 1.261  \\ \hline
13.048 & 6.842  & 0.510 & 0.179 & 0.513 & 0.409 & -2.087 & 1.005  \\ \hline
34.881 & 20.899 & 0.577 & 0.191 & 0.931 & 0.756 & -2.705 & 1.373  \\ \hline
46.652 & 26.360 & 0.585 & 0.207 & 0.901 & 0.707 & 1.248  & -3.409 \\ \hline
\end{tabular}
\caption{CycleGAN Ergebnisse unter verschiedenen Hyperparameter Teil2}
\label{evaluation:cycleGan_table2}
\end{table}


Im Kontrast dazu bleibt die Genauigkeit der Diskriminatoren für Trainingsdaten stabil bei etwa $50\%$. Jedoch zeigen sich erhebliche Schwankungen je nach Epoche für die Testdaten, insbesondere für $D_Y$. Dies legt nahe, dass die Diskriminatoren möglicherweise Schwierigkeiten haben, mit neuen, nicht trainierten Daten umzugehen. Diese Schwankungen könnten auf Überanpassung oder eine mangelnde Generalisierungsfähigkeit des Modells hinweisen, trotz der implementierten Maßnahmen wie dem Hinzufügen von Rauschen und Dropout-Schichten (\ref{evaluation:cycleGan_Metriken}).
\\\newline
Die Laufzeit des Trainings spiegelt die Komplexität der Modelle wider, insbesondere auf der GPU, wo jede Epoche im Durchschnitt etwa 2 Minuten dauert. Im Vergleich dazu benötigt die CPU, selbst nach einer Reduzierung der Residualblöcke auf einen Block, rund 6 Minuten pro Epoche. Die Nutzung von Instanznormalisierung trägt zur beschleunigten Konvergenz und Verbesserung der Laufzeit bei. Zudem zeigt sich, dass die Verwendung von Residualblöcken eine bessere Laufzeit ermöglicht im Vergleich zu Generatoren mit zusätzlichen Schichten. 
\\\newline
Insgesamt zeigt die Evaluation, dass das CycleGAN-Modell trotz einiger Herausforderungen in der Generierung von detaillierten Strukturen und der Generalisierung auf neue Daten vielversprechende Ergebnisse erzielt.

