\chapter{Problembeschreibung}
Generative Adversarial Networks (GANs) haben in den letzten Jahren erhebliche Aufmerksamkeit in der Forschung und Industrie erlangt. Diese neuartige Klasse von künstlichen neuronalen Netzwerken hat das Potenzial, realistische Daten zu generieren und komplexe Probleme in verschiedenen Domänen zu lösen. Im Rahmen dieser Arbeit liegt der Fokus auf zwei spezifischen GAN-Varianten: Pix2Pix und CycleGAN. Beide Ansätze sind auf die Generierung von Bildern ausgerichtet und haben in der Bildverarbeitung und Computer Vision Anwendung gefunden.
\\
Pix2Pix konzentriert sich auf die direkte Zuordnung zwischen Eingabe- und Ausgabebildern, während CycleGAN die Fähigkeit besitzt, nicht paarweise zugeordnete Datensätze zu übersetzen. Diese Modelle haben das Potenzial, in verschiedenen Szenarien wie der Stilübertragung, der Bildsegmentierung und der Domänenanpassung verwendet zu werden. Trotz ihrer vielversprechenden Anwendungen gibt es jedoch verschiedene Herausforderungen im Design, in der Implementierung und in der Analyse dieser Modelle. Es stellt sich die Frage, wie die Modelle effektiv gestaltet werden können, um optimale Leistung zu erzielen und welche Strategie am besten geeignet sind, um die Modelle erfolgreich zu trainieren und zu evaluieren. 
\subsection{Herausforderungen im Design}
Die Gestaltung von GANs, insbesondere von Pix2Pix und CycleGAN, ist mit verschiedenen Herausforderungen verbunden. Die Architekturwahl, die Hyperparameterabstimmung und die Integration von Regularisierungstechniken sind kritische Aspekte, die sorgfältig berücksichtigt werden müssen. Die Auswirkungen dieser Entscheidungen auf die Leistung und Konvergenz der Modelle sind nicht trivial und erfordern eine eingehende Analyse.
\subsection{Herausforderungen in der Implementierung}
Die erfolgreiche Implementierung von Pix2Pix und CycleGAN erfordert die Berücksichtigung von Aspekten wie Datenpräparation, Training und Evaluierung. Die Auswahl geeigneter Datensätze, die Handhabung von Ungleichgewichten in den Daten, die Optimierung von Trainingsparametern und die Vermeidung von Overfitting sind entscheidende Schritte. Es ist von entscheidender Bedeutung, diese Implementierungsherausforderungen zu verstehen und zu bewältigen, um die Modelle effektiv nutzen zu können.
\subsection{Herausforderungen in der Analyse}
Die Analyse von Pix2Pix und CycleGAN beinhaltet die Bewertung ihrer Generierungsfähigkeiten, die Quantifizierung von Artefakten in den generierten Bildern und die Untersuchung von Konvergenzproblemen während des Trainings. 
\subsection{Forschungs}