\chapter{Problembeschreibung}
Generative Adversarial Networks (GANs) haben in den letzten Jahren erhebliche Aufmerksamkeit in der Forschung und Industrie erlangt. Diese neuartige Klasse von künstlichen neuronalen Netzwerken hat das Potenzial, realistische Daten zu generieren und komplexe Probleme in verschiedenen Domänen zu lösen. Im Rahmen dieser Arbeit liegt der Fokus auf zwei spezifischen GAN-Varianten: Pix2Pix und CycleGAN. 
\\
Pix2Pix konzentriert sich auf die direkte Zuordnung zwischen Eingabe- und Ausgabebildern, während CycleGAN die Fähigkeit besitzt, nicht paarweise zugeordnete Datensätze zu übersetzen. 
Diese Modelle sind auf die Generierung von Bildern ausgerichtet und haben das Potenzial, in verschiedenen Szenarien wie der Stilübertragung, der Bildsegmentierung und der Domänenanpassung verwendet zu werden. Trotz ihrer vielversprechenden Anwendungen gibt es jedoch verschiedene Herausforderungen im Design, in der Implementierung und in der Analyse dieser Modelle. Es stellt sich die Frage, wie die Modelle effektiv gestaltet werden können, um optimale Leistung zu erzielen und welche Strategie am besten geeignet sind, um die Modelle erfolgreich zu trainieren und zu evaluieren. 

\subsubsection{Herausforderungen im Design}
Die Entwicklung von GAN, insbesondere von Modellen wie Pix2Pix und CycleGAN, ist mit zahlreichen Herausforderungen verbunden. Die Wahl der Architektur, die Anpassung der Hyperparameter und die Integration von Regularisierungstechniken sind entscheidende Aspekte, die mit besonderer Sorgfalt angegangen werden müssen. Diese Herausforderungen haben einen großen Einfluss auf die Fähigkeit der Modelle, realistische und generalisierte Ergebnisse zu liefern.
\\
Die Wahl der Architektur spielt eine zentrale Rolle und wirkt sich direkt auf die Fähigkeit des Modells aus, komplexe Transformationen und Generierungsaufgaben durchzuführen. Die Anpassung der Hyperparameter erfordert eine Feinabstimmung, um die Konvergenz des Modells ohne Überanpassung zu gewährleisten. Die Integration von Regularisierungstechniken ist von wesentlicher Bedeutung, um das Modell vor Überanpassung zu schützen und seine Gesamtleistung zu verbessern.
\\
Die Auswirkungen dieser Entscheidungen auf die Leistung und Konvergenz der Modelle sind daher nicht unbedeutend und erfordern eine eingehende Analyse, um sicherzustellen, dass die GANs in der Lage sind, qualitativ hochwertige und realistische Ergebnisse zu liefern und gleichzeitig eine stabile Konvergenz während des Trainings zu gewährleisten.

\subsubsection{Herausforderungen in der Implementierung}
Die Auswahl geeigneter Datensätze, der Umgang mit Datenungleichgewichten, die Optimierung der Trainingsparameter und die Vermeidung von Overfitting sind wichtige Schritte bei der Implementierung von Pix2Pix und CycleGAN. Dies umfasst die Datenaufbereitung, das Training und die Evaluierung der Modelle. 
\\
Die Auswahl des Datensatzes hat einen großen Einfluss auf die Fähigkeit des Modells, realistische Ergebnisse zu liefern. Dabei ist nicht nur die Menge, sondern auch die Vielfalt der Daten von Bedeutung. Auch der Umgang mit Ungleichgewichten in den Daten ist von hoher Relevanz, um sicherzustellen, dass das Modell nicht in Richtung bestimmter Merkmale verzerrt wird.
\\
Die Optimierung der Trainingsparameter, einschließlich der Lernraten und der Batchgrößen, ist ein Feinabstimmungsprozess, um eine stabile Konvergenz des Modells zu gewährleisten. Gleichzeitig ist es von entscheidender Notwendigkeit, Overfitting durch die Implementierung geeigneter Regularisierungstechniken zu vermeiden.
\\
Das Verständnis und die zielgerichtete Bewältigung dieser Implementierungsherausforderungen sind entscheidend, um sicherzustellen, dass Pix2Pix und CycleGAN effektiv in verschiedenen Anwendungsbereichen eingesetzt werden können. Durch eine gründliche Untersuchung dieser Aspekte können Modelle entwickelt werden, die nicht nur leistungsfähig, sondern auch robust und generalisierbar sind.

\subsubsection{Herausforderungen in der Analyse}
Die Analyse von Pix2Pix und CycleGAN umfasst mehrere Schlüsselaspekte, die zum Verständnis der Leistung und Zuverlässigkeit dieser GANs beitragen. Dazu gehören die Bewertung der Generierungsfähigkeiten, die Quantifizierung von Artefakten und die Untersuchung von Knovergenzproblemen. 
\\
Die Evaluierung umfasst qualitative Bewertungen der von den Modellen erzeugten Bilder. Dazu können visuelle Inspektionen und Vergleiche mit den Originaldaten durchgeführt werden. Unerwünschte Muster und Unvollkommenheiten können ebenfalls visuell oder mit Hilfe von Metriken identifiziert werden. Dazu gehören Messungen wie das Signal-Rausch-Verhältnis (SNR), der strukturelle Ähnlichkeitsindex (SSI) oder der Interception Score (IS). \\ Die Analyse kann ferner die Überwachung von Verlustkurven umfassen, um Konvergenzprobleme während des Trainings zu untersuchen. 
\\
Analyse und Bewertung sind entscheidend, um die Modelle weiter zu verbessern, ihre Anwendbarkeit zu erweitern und sicherzustellen, dass sie ihren Zweck erfüllen.