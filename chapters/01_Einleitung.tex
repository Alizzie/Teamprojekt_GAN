\chapter{Einleitung}
In den letzten Jahren haben sich Generative Adversarial Networks (GANs), eine bahnbrechende Entwicklung in der Welt der künstlichen Intelligenz, die maßgeblich auf die Arbeit von Ian Goodfellow und seinem Team zurückgeht, als revolutionäre Methode etabliert. Diese technologische Entwicklung, die erstmals 2014 in einem wegweisenden Paper vorgestellt wurde, hat es ermöglicht, dass Maschinen in der Bildverarbeitung Leistungen erbringen, die früher undenkbar waren. GANs ursprünglich konzipiert als Instrumente zur Erstellung von realistischen Bildern, haben weitreichende Anwendungen in verschiedenen Bereichen gefunden, insbesondere in der Bildverarbeitung, wo sie durch ihre Fähigkeit, Bilder zu generieren und zu transformieren, eine neue Ära eingeleitet haben. \newline
Die innovative Struktur von GANs besteht aus zwei gegeneinander antretenden Netzwerken: dem Generator und dem Diskriminator. Der Generator hat die Aufgabe, Daten zu erzeugen, die von realen Daten ununterscheidbar sind, während der Diskriminator die Echtheit dieser Daten beurteilt. Durch diesen Wettbewerb lernt der Generator, immer bessere Fälschungen zu erzeugen und der Diskriminator wird effizienter im Erkennen dieser Fälschungen. Dieses Zusammenspiel führt zu einer stetigen Verbesserung beider Fälschungen. Dieses Zusammenspiel führt zu einer stetigen Verbesserung beider Netzwerke und ermöglicht die Erstellung hochqualitativer generierter Daten. Diese Methodik bildet den Kern der Funktionsweise von GANs und ist entscheidend für ihre Anwendung in der Bildverarbeitung, wo sie genutzt werden, um realistische Bilder zu generieren und existierende Bilder in vielfältiger Weise zu transformieren \cite{Goodfellow2014}.
\newline
Aufbauend auf dieser grundlegenden Architektur von GANs, hat sich das spezialisierte Framework Pix2Pix als bedeutende Weiterentwicklung in der Welt der Bild-zu-Bild-Übersetzung etabliert. Entwickelt von Phillip Isola und seinem Forschungsteam, adaptiert Pix2Pix das GAN-Prinzip für spezifische Anforderungen der Bildtransformation. Während der klassische Ansatz von GANs auf die Erzeugung neuer, realistisch wirkender Daten abzielt, fokussiert Pix2Pix auf die präzise Übersetzung von Eingangsbildern in gewünschte Ausgangsbilder. Hierbei übernimmt der Generator die Rolle der Bildtransformation, indem er relevante Merkmale aus dem Eingangsbild extrahiert und diese in ein neues, zielgerichtetes Bild umwandelt. Der Diskriminator bewertet anschließend diese Übersetzung, wodurch die Genauigkeit und Realitätsnähe der Bildtransformation weiter optimiert wird. Diese spezialisierte Anwendung von GANs demonstriert die Vielseitigkeit und Anpassungsfähigkeit der Technologie und öffnet neue Horizonte in der Bildverarbeitung \cite{PhillipIsola.}.
\newline
CycleGAN, entwickelt von Jun-Yan Zhu und Kollegen, ist eine Erweiterung der Generative Adversarial Networks (GANs) und speziell für die Aufgabe der Bild-zu-Bild-Übersetzung in Fällen konzipiert, wo keine paarigen Trainingsdaten vorhanden sind. Es unterscheidet sich von herkömmlichen GANs und Pix2Pix durch seinen innovativen Ansatz der "Zykluskonsistenz".
Im Gegensatz zu Pix2Pix, das paarige Trainingsdaten benötigt (wo jedes Eingangsbild einem spezifischen Ausgangsbild zugeordnet ist), ermöglicht CycleGAN die Übersetzung zwischen zwei Bildsammlungen, ohne dass eine Eins-zu-eins-Beziehung zwischen den Bildern in den Sammlungen besteht. Diese Fähigkeit ist besonders wertvoll für Aufgaben wie das Umwandeln von Sommerbildern in Winterbilder oder das Übertragen von Stilen zwischen verschiedenen Künstlern, wo paarige Trainingsdaten schwierig oder unmöglich zu sammeln sind.
Die Schlüsselinnovation von CycleGAN ist die Einführung einer Zykluskonsistenzverlust-Funktion. Diese Funktion sorgt dafür, dass ein Eingangsbild, das in ein Bild einer anderen Domain übersetzt und dann zurück in die ursprüngliche Domain übersetzt wird, dem Originalbild ähnlich bleibt. Zum Beispiel, wenn ein Foto eines Pferdes in ein Zebra umgewandelt wird und dann wieder zurück in ein Pferd, sollte das endgültige Bild dem ursprünglichen Pferdefoto ähnlich sein. Diese Zykluskonsistenz hilft, bedeutungsvolle und kohärente Übersetzungen zwischen unverbundenen Bildsammlungen zu gewährleisten \cite{Zhu.2017}.\newline
In dieser Arbeit wird zunächst ein Überblick über verschiedene GAN-Architekturen gegeben, einschließlich Pix2Pix und CycleGAN, und deren Anwendungen in der Bildverarbeitung untersucht. Darauf folgend wird eine detaillierte Beschreibung der Implementierung verschiedener GAN-Architekturen präsentiert, einschließlich der Behandlung von Trainingsdaten, der Architektur und der Optimierungstechniken. \newline
Der Einsatz von GANs in der Bildverarbeitung bietet vielversprechende Ergebnisse, jedoch unterscheidet sich der Stil und die Herangehensweise dieser Technologie von traditionellen Bildverarbeitungsmethoden. Diese Arbeit zielt darauf ab, die Möglichkeiten von Pix2Pix- und Cycle-GAN in der Bildverarbeitung zu ergründen indem diese Modelle selbst Implementiert und Trainiert werden. \newline
Abschließend werden die Ergebnisse der Durchführung des GAN-Trainings präsentiert und evaluiert. Es wird zudem auch noch untersucht welchen Einfluss die Hyperparameter auf das GAN-Training haben um somit die Resultate des Trainings weiter zu verbessern. \newline 
Im Anschluss wird diese Arbeit mit einer Diskussion über die erzielten Ergebnisse und einer Bewertung der verschiedenen GAN-Architekturen abgeschlossen. Ein besonderer Fokus liegt auf der praktischen Anwendung dieser Technologien in der Bildverarbeitung und den Möglichkeiten, die sich daraus für zukünftige Forschungen und Entwicklungen ergeben. \newline 
