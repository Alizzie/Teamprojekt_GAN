\chapter{Evaluation}
\section{Bewertungskriterien}
\subsection{Diskriminator-Verlust}
Der Diskriminator-Verlust ist ein wesentliches Maß für die Effektivität des Diskriminators in einem GAN-Modell. Ein niedriger Diskriminator-Verlust bedeutet, dass der Diskriminator erfolgreich zwischen echten und generierten Bildern unterscheiden kann. Im Kontext des Pix2Pix GANs zeigt ein abnehmender Diskriminator-Verlust über die Epochen hinweg, dass der Diskriminator zunehmend besser darin wird, die Authentizität der Bilder zu beurteilen. Ein hoher Diskriminator-Verlust weist hingegen darauf hin, dass das Modell Schwierigkeiten hat, echte von generierten Bildern zu unterscheiden, was ein Indikator für Verbesserungspotenzial im Training oder der Modellarchitektur sein könnte.

\subsection{Generator-Verlust}
Der Generator-Verlust ist ein Indikator für die Fähigkeit des Generators, Bilder zu erzeugen, die vom Diskriminator als echt eingestuft werden. Ein hoher Generator-Verlust deutet darauf hin, dass der Generator die realen Daten noch nicht effektiv nachahmen kann, während ein niedriger Generator-Verlust ein Zeichen dafür ist, dass die generierten Bilder den echten immer ähnlicher werden. Im Verlauf des Trainings sollte der Generator-Verlust tendenziell abnehmen, was darauf hindeutet, dass der Generator lernt, überzeugendere Bilder zu produzieren.

\subsection{Diskriminator Genauigkeit}
Die Diskriminator-Genauigkeit gibt an, wie oft der Diskriminator korrekt zwischen echten und gefälschten Bildern unterscheidet. Eine hohe Genauigkeit bedeutet, dass der Diskriminator effektiv arbeitet, während eine niedrige Genauigkeit auf Probleme bei der Unterscheidung hinweisen kann. Es ist wichtig, ein Gleichgewicht zu finden, denn eine zu hohe Genauigkeit kann darauf hindeuten, dass der Diskriminator die generierten Bilder zu leicht erkennt, was auf ein Problem mit dem Generator hinweisen könnte.\newline
In einem ideal funktionierenden GAN sollte der Diskriminator eine Genauigkeit von etwa 50$\%$ erreichen. Das bedeutet, dass der Diskriminator die echten von dem generierten Bildern nur zufällig unterscheiden bzw. nicht mehr zuverlässig unterscheiden kann, was zeigt dass das GAN gut trainiert wurde.

\subsection{SSIM-Score (Structural Similarity Index)}
Der SSIM-Score ist ein Maß für die visuelle Ähnlichkeit zwischen den generierten Bildern und den entsprechenden Originalbildern. Er bewertet die Bildqualität anhand von Faktoren wie Helligkeit, konstrast und Struktur. Ein hoher SSIM-Wert deutet darauf in, dass die generierten Bilder den echten Bildern sehr ähnlich sind, was ein Zeichen für eine hohe Bildqualität ist. Ein niedrieger SSIM-Wert kann auf deutliche Unterschiede in der visuellen Struktur hinweisen, was Verbesserungsbedarf im Generator-Training oder in der Modellarchitektur signalisiert.

\section{Pix2Pix: Ergebnisse und objektive Bewertung}
 Diese Flexibilität in der Kanalverarbeitung ermöglicht eine breitere Anwendung des Pix2Pix-Modells auf verschiedene Bildtypen. Die Bilder müssen nicht in einer bestimmten Weise vorverarbeitet werden, da das Modell direkt auf den Rohpixeln arbeitet. Diese Flexibilität in der Kanalverarbeitung und die Fähigkeit, direkt auf Rohpixeln zu arbeiten, unterstreichen die Vielseitigkeit des Pix2Pix-Modells.
  Durch empirische Tests hat sich herausgestellt das eine initiale Lernrate von 0.0002 und die Momentum-Parameter von $\beta1$ = 0.5 und $\beta$ = 0.999 optimal sind, um die Balance zwischen Lerngeschwindigkeit und Stabilität des Trainingsprozesses zu optimieren. Auch die Wahl einer kleinen Batchgröße, typischerweise 1, spielt eine entscheidende Rolle, um die Trainingseffizienz zu maximieren und qualitativ hochwertige Ergebnisse zu erzielen. Diese spezifischen Einstellungen der Trainingsparameter tragen maßgeblich dazu bei, das Potenzial des Pix2Pix-Modells voll auszuschöpfen.\cite{PhillipIsola.}.\newline
  Für den Generator und den Diskriminator, wird der Adam-Optimierer mit einer Lernrate von 0.0002 und den Momentum-Parametern $\beta$1 = 0.5 und $\beta$2 = 0.999 verwendet. Diese Einstellungen einen guten Kompromiss zwischen der Lerngeschwindigkeit und der Stabilität des Trainingsprozesses zu finden. Die Batchgröße ist im Code auf 1 gesetzt. Eine kleine Batchgröße kann zu einer höheren Stabilität im Trainingsprozess beitragen. Der Hyperparameter $LAMBDA$ wird verwendet, um das Gewicht des L1-Verlustes im Generatorverlust zu steuern. Ein hoher Wert von $LAMBDA$ betont die Bedeutung der Inhaltsähnlichkeit zwischen den generierten und den Zielbildern. Die Anzahl der Trainingsepochen ist auf 450 gesetzt, was darauf hindeutet, dass das Modell eine umfangreiche Trainingsdauer durchläuft, um eine optimale Leistung zu erreichen.
\section{CycleGAN: Ergebnisse und objekte Bewertung}
\section{Vergleich und Bewertung von Pix2Pix und CycleGAN}
\section{Vergleich von Pix2Pix und CycleGAN}
- Matching Paare von Bildern sind ebenfalls für das Training nicht nötig (crewall)
- Macht die Datenvorbereitung einfacher und öffnet neue Techniken für Applikationen (crewall)