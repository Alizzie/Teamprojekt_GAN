\chapter{Evaluation}
\section{Bewertungskriterien}
\subsection{Diskriminator-Verlust}
Der Diskriminator-Verlust ist ein wesentliches Maß für die Effektivität des Diskriminators in einem GAN-Modell. Ein niedriger Diskriminator-Verlust bedeutet, dass der Diskriminator erfolgreich zwischen echten und generierten Bildern unterscheiden kann. Im Kontext eines GANs zeigt ein abnehmender Diskriminator-Verlust über die Epochen hinweg, dass der Diskriminator zunehmend besser darin wird, die Authentizität der Bilder zu beurteilen. Ein hoher Diskriminator-Verlust weist hingegen darauf hin, dass das Modell Schwierigkeiten hat, echte von generierten Bildern zu unterscheiden, was ein Indikator für Verbesserungspotenzial im Training oder der Modellarchitektur sein könnte.

\subsection{Generator-Verlust}
Der Generator-Verlust ist ein Indikator für die Fähigkeit des Generators, Bilder zu erzeugen, die vom Diskriminator als echt eingestuft werden. Ein hoher Generator-Verlust deutet darauf hin, dass der Generator die realen Daten noch nicht effektiv nachahmen kann, während ein niedriger Generator-Verlust ein Zeichen dafür ist, dass die generierten Bilder den echten immer ähnlicher werden. Im Verlauf des Trainings sollte der Generator-Verlust tendenziell abnehmen, was darauf hindeutet, dass der Generator lernt, überzeugendere Bilder zu produzieren.

\subsection{Diskriminator Genauigkeit}
Die Diskriminator-Genauigkeit gibt an, wie oft der Diskriminator korrekt zwischen echten und gefälschten Bildern unterscheidet. Eine hohe Genauigkeit bedeutet, dass der Diskriminator effektiv arbeitet, während eine niedrige Genauigkeit auf Probleme bei der Unterscheidung hinweisen kann. Es ist wichtig, ein Gleichgewicht zu finden, denn eine zu hohe Genauigkeit kann darauf hindeuten, dass der Diskriminator die generierten Bilder zu leicht erkennt, was auf ein Problem mit dem Generator hinweisen könnte.\newline
In einem ideal funktionierenden GAN sollte der Diskriminator eine Genauigkeit von etwa 50$\%$ erreichen. Das bedeutet, dass der Diskriminator die echten von dem generierten Bildern nur zufällig unterscheiden bzw. nicht mehr zuverlässig unterscheiden kann, was zeigt dass das GAN gut trainiert wurde.

\subsection{SSIM-Score (Structural Similarity Index)}
Der SSIM-Score ist ein Maß für die visuelle Ähnlichkeit zwischen den generierten Bildern und den entsprechenden Originalbildern. Er bewertet die Bildqualität anhand von Faktoren wie Helligkeit, Kontrast und Struktur. Ein hoher SSIM-Wert deutet darauf in, dass die generierten Bilder den echten Bildern sehr ähnlich sind, was ein Zeichen für eine hohe Bildqualität ist. Ein niedriger SSIM-Wert kann auf deutliche Unterschiede in der visuellen Struktur hinweisen, was Verbesserungsbedarf im Generator-Training oder in der Modellarchitektur signalisiert.

\section{Pix2Pix: Ergebnisse und objektive Bewertung}
\subsubsection{Evaluierung der Trainingsresultate nach 100 Epochen}
Die Evaluierung der Trainingsresultate des Pix2Pix-Modells basiert auf einer Analyse der 100 durchgeführten Trainingsepochen. Während dieser Epochen wurden spezifische Metriken wie Generator- und Diskriminator-Verluste sowie Diskriminator-Genauigkeit und der Structural Similarity Index Measure (SSIM) überwacht. \newline
Der Generatorverlust zeigte eine anfängliche Tendenz zur Reduktion von 13.782 auf 10.581, was auf eine erfolgreiche Anpassung des Modells an die Trainingsdaten hindeutet. Parallel dazu reduzierte sich der Diskriminatorverlust von 1.009 auf 0.323, was auf eine effiziente Differenzierung zwischen realen und generierten Bildern durch den Diskriminator hinweist.
\begin{figure}[h]
	\centering
	\includegraphics[width=1.0\textwidth]{images/Pix2PixResults/Verlust0-100.png}
	\caption{Generator- und Diskriminatorverlust des 1. Trainingsdurchlaufs mit 100 Epochen, eigene Erstellung mittels matplotlib.pyplot-Bibliothek }
	\label{fig:Verlustkurve 0-100}
\end{figure}

Die Genauigkeit des Diskriminators ist von anfänglich etwa 75$\%$ auf ungefähr 94$\%$ in der letzten Epoche gestiegen. Dies zeigt, dass der Diskriminator im Verlauf des Trainingsprozesses besser darin wurde, echte Bilder von den vom Generator erzeugten Bildern zu unterscheiden.
\begin{figure}[h]
	\centering
	\includegraphics[width=1.0\textwidth]{images/Pix2PixResults/Genauigkeit0-100.png}
	\caption{Diskriminatorgenauigkeit des 1. Trainingsdurchlaufs mit 100 Epochen, eigene Erstellung mittels matplotlib.pyplot-Bibliothek }
	\label{fig:Genauigkeit 0-100}
\end{figure}

Der SSIM-Wert, der die Bildqualität misst, begann bei 0.553 und endete bei 0.656. Diese kontinuierliche Verbesserung deutet darauf hin, dass die generierten Bilder im Laufe des Trainings eine zunehmende Ähnlichkeit mit den Zielbildern erreicht haben.
\begin{figure}[h]
	\centering
	\includegraphics[width=0.95\textwidth]{images/Pix2PixResults/SSIM0-100.png}
	\caption{SSIM-Score des 1. Trainingsdurchlaufs mit 100 Epochen, eigene Erstellung mittels matplotlib.pyplot-Bibliothek}
	\label{fig:SSIM 0-100}
\end{figure}

\newpage
\section{CycleGAN: Ergebnisse und objekte Bewertung}
\section{Vergleich und Bewertung von Pix2Pix und CycleGAN}

\section{Ergebnisse und objektive Bewertung}
In diesem Abschnitt liegt der Fokus auf der objektiven Bewertung der Leistung des CycleGAN-Algorithmus. Es werden verschiedene Metriken und Kriterien herangezogen, um eine umfassende Bewertung durchzuführen, einschließlich der Verluste der Generatoren und Diskriminatoren bei Variation der Hyperparameter. Darüber hinaus wird eine detaillierte Analyse der Genauigkeit der Diskriminatoren und des SSIM-Scores durchgeführt. Diese Messungen sind sowohl für die genaue Bestimmung der Qualität der generierten Ergebnisse als auch für die Bewertung der Lernfähigkeit der Modelle von Bedeutung.

Die verwendeteten Datensätze für die Evaluierung stammen aus dem Berkeley-Repository. Besondere Beachtung wird dem 'maps'-Datensatz geschenkt, bei dem Transformationen zwischen Kartenansichten und Satellitenbildern durchgeführt wurden. Das Training wurde mehrmals mit verschiedenen Hyperparameter-\\Konfigurationen durchgeführt, wobei Parameter wie der Lambda-Wert, die Anzahl der Epochen und die Lernrate des Optimizers variiert wurden. Ziel ist es, eine Konfiguration zu finden, bei der die Qualität der erzeugten Bilder nahe an der Qualität des Originals und die Leistung des Diskriminators optimal ist.

Die subjektive Bewertung erfolgt durch die visuelle Beurteilung der generierten Bilder aus dem Testdatensatz hinsichtlich stimmiger Farben, Struktur und Rauschen im Vergleich zu den Originalbildern. Darauf aufbauend erfolgt eine objektive Bewertung mithilfe der definierten Metriken.


\subsection{Pix2Pix}
 Diese Flexibilität in der Kanalverarbeitung ermöglicht eine breitere Anwendung des Pix2Pix-Modells auf verschiedene Bildtypen. Die Bilder müssen nicht in einer bestimmten Weise vorverarbeitet werden, da das Modell direkt auf den Rohpixeln arbeitet. Diese Flexibilität in der Kanalverarbeitung und die Fähigkeit, direkt auf Rohpixeln zu arbeiten, unterstreichen die Vielseitigkeit des Pix2Pix-Modells.
  Durch empirische Tests hat sich herausgestellt das eine initiale Lernrate von 0.0002 und die Momentum-Parameter von $\beta1$ = 0.5 und $\beta$ = 0.999 optimal sind, um die Balance zwischen Lerngeschwindigkeit und Stabilität des Trainingsprozesses zu optimieren. Auch die Wahl einer kleinen Batchgröße, typischerweise 1, spielt eine entscheidende Rolle, um die Trainingseffizienz zu maximieren und qualitativ hochwertige Ergebnisse zu erzielen. Diese spezifischen Einstellungen der Trainingsparameter tragen maßgeblich dazu bei, das Potenzial des Pix2Pix-Modells voll auszuschöpfen.\cite{PhillipIsola.}.\newline
  Für den Generator und den Diskriminator, wird der Adam-Optimierer mit einer Lernrate von 0.0002 und den Momentum-Parametern $\beta$1 = 0.5 und $\beta$2 = 0.999 verwendet. Diese Einstellungen einen guten Kompromiss zwischen der Lerngeschwindigkeit und der Stabilität des Trainingsprozesses zu finden. Die Batchgröße ist im Code auf 1 gesetzt. Eine kleine Batchgröße kann zu einer höheren Stabilität im Trainingsprozess beitragen. Der Hyperparameter $LAMBDA$ wird verwendet, um das Gewicht des L1-Verlustes im Generatorverlust zu steuern. Ein hoher Wert von $LAMBDA$ betont die Bedeutung der Inhaltsähnlichkeit zwischen den generierten und den Zielbildern. Die Anzahl der Trainingsepochen ist auf 450 gesetzt, was darauf hindeutet, dass das Modell eine umfangreiche Trainingsdauer durchläuft, um eine optimale Leistung zu erreichen.
<<<<<<< HEAD
<<<<<<< HEAD
>>>>>>> 6d9a1f1fc2f012db8cd1a494c61466a617f0b840
=======
>>>>>>> 6d9a1f1fc2f012db8cd1a494c61466a617f0b840
=======
>>>>>>> 6d9a1f1fc2f012db8cd1a494c61466a617f0b840

\subsection{CycleGAN}
\section{CycleGAN: Ergebnisse und objekte Bewertung}

In diesem Abschnitt liegt der Fokus auf der objektiven Bewertung der Leistung des CycleGAN-Algorithmus. Es werden verschiedene Metriken und Kriterien herangezogen, um eine umfassende Bewertung durchzuführen, einschließlich der Verluste der Generatoren und Diskriminatoren bei Variation der Hyperparameter. Darüber hinaus wird eine detaillierte Analyse der Genauigkeit der Diskriminatoren und des SSIM-Scores durchgeführt. Diese Messungen sind sowohl für die genaue Bestimmung der Qualität der generierten Ergebnisse als auch für die Bewertung der Lernfähigkeit der Modelle von Bedeutung.

Die verwendeteten Datensätze für die Evaluierung stammen aus dem Berkeley-Repository. Besondere Beachtung wird dem 'maps'-Datensatz geschenkt, bei dem Transformationen zwischen Kartenansichten und Satellitenbildern durchgeführt wurden. Das Training wurde mehrmals mit verschiedenen Hyperparameter-\\Konfigurationen durchgeführt, wobei Parameter wie der Lambda-Wert, die Anzahl der Epochen und die Lernrate des Optimizers variiert wurden. Ziel ist es, eine Konfiguration zu finden, bei der die Qualität der erzeugten Bilder nahe an der Qualität des Originals und die Leistung des Diskriminators optimal ist.

Die subjektive Bewertung erfolgt durch die visuelle Beurteilung der generierten Bilder aus dem Testdatensatz hinsichtlich stimmiger Farben, Struktur und Rauschen im Vergleich zu den Originalbildern. Darauf aufbauend erfolgt eine objektive Bewertung mithilfe der definierten Metriken.
\\\newline
Seien die Datensätze $X: Maps$ und $Y: Sateliten$, $G: X\rightarrow Y$ und $F: Y\rightarrow X$ die Generatoren sowie $D_X$ und $D_Y$ die Diskriminatoren, die jeweils zwischen Bilder in der Domäne $X$ und $Y$ unterscheiden. 
\\\newline

In allen Durchläufen zeigt sich, dass die Generatoren Probleme mit feinen und detailreichen Details haben. Weil die meisten Trainingsbilder Straßen und Häuser zeigen, können die Generatoren besser die Straßenlinien erzeugen.

Bilder mit einer größeren Vielfalt wie Flüsse und Wälder sind eine größere Herausforderung für die Generatoren. Die Erstellung der Karten gestaltet sich schwierig in Bezug auf die Struktur und Farbkodierung. Oft werden diese Flächen in derselben Farbe wie die Straßen erzeugt, und es wird versucht, auf der Grundlage dieser großen Freiflächen kleine Straßen zu erzeugen. Bei allen Versuchen gab es Probleme bei der Erfassung feiner Strukturen in den Wäldern sowie bei der einheitlichen Farbkodierung der Flüsse in den Satellitenbildern. 
\\\newline
Es stellt sich heraus, dass für den Lambda-Wert die Größen 0.0002 die besten Ergebnisse erzielen hinsichtlich des SSIM-Scores und der Test-Genauigkeiten bei den Diskriminatoren. 
Die Genauigkeit

\begin{figure}
    \begin{subfigure}[t]{.4\textwidth}
      \centering
      \includegraphics[width=\linewidth]{example-image-a}
      \caption{\textbf{Schnitt}: $A \cup B$: Element liegt in $A$ \textbf{oder} in $B$.}
    \end{subfigure}
    \hfill
    \begin{subfigure}[t]{.4\textwidth}
      \centering
      \includegraphics[width=\linewidth]{example-image-b}
      \caption{\textbf{Vereinigung}: $A \cap B$: Element liegt in $A$ \textbf{und} in $B$.}
    \end{subfigure}
  
    \medskip
  
    \begin{subfigure}[t]{.4\textwidth}
      \centering
      \includegraphics[width=\linewidth]{example-image-c}
      \caption{\textbf{Differenz}: $A \setminus B$: Element liegt in $A$ \textbf{nicht} in $B$. (\textit{A ohne B})}
    \end{subfigure}
    \hfill
    \begin{subfigure}[t]{.4\textwidth}
      \centering
      \includegraphics[width=\linewidth]{example-image-a}
      \caption{\textbf{Symmetrische Differenz}: $A \Delta B$: Element liegt \textbf{entweder} in $A$ oder in $B$.}
    \end{subfigure}
  \end{figure}


  Die Komplexität der Modelle spiegelt sich vor allem in der Laufzeit des Trainings wieder. Diese umfasst im Durchschnitt 2 Minuten pro Epoche auf der GPU, während die CPU 
\section{Vergleich von Pix2Pix und CycleGAN}
- Matching Paare von Bildern sind ebenfalls für das Training nicht nötig (crewall)
- Macht die Datenvorbereitung einfacher und öffnet neue Techniken für Applikationen (crewall)