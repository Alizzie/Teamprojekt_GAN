\chapter{Evaluation}
\section{Bewertungskriterien}
\section{Pix2Pix: Ergebnisse und objektive Bewertung}
 Diese Flexibilität in der Kanalverarbeitung ermöglicht eine breitere Anwendung des Pix2Pix-Modells auf verschiedene Bildtypen. Die Bilder müssen nicht in einer bestimmten Weise vorverarbeitet werden, da das Modell direkt auf den Rohpixeln arbeitet. Diese Flexibilität in der Kanalverarbeitung und die Fähigkeit, direkt auf Rohpixeln zu arbeiten, unterstreichen die Vielseitigkeit des Pix2Pix-Modells.
  Durch empirische Tests hat sich herausgestellt das eine initiale Lernrate von 0.0002 und die Momentum-Parameter von $\beta1$ = 0.5 und $\beta$ = 0.999 optimal sind, um die Balance zwischen Lerngeschwindigkeit und Stabilität des Trainingsprozesses zu optimieren. Auch die Wahl einer kleinen Batchgröße, typischerweise 1, spielt eine entscheidende Rolle, um die Trainingseffizienz zu maximieren und qualitativ hochwertige Ergebnisse zu erzielen. Diese spezifischen Einstellungen der Trainingsparameter tragen maßgeblich dazu bei, das Potenzial des Pix2Pix-Modells voll auszuschöpfen.\cite{PhillipIsola.}.\newline
  Für den Generator und den Diskriminator, wird der Adam-Optimierer mit einer Lernrate von 0.0002 und den Momentum-Parametern $\beta$1 = 0.5 und $\beta$2 = 0.999 verwendet. Diese Einstellungen einen guten Kompromiss zwischen der Lerngeschwindigkeit und der Stabilität des Trainingsprozesses zu finden. Die Batchgröße ist im Code auf 1 gesetzt. Eine kleine Batchgröße kann zu einer höheren Stabilität im Trainingsprozess beitragen. Der Hyperparameter $LAMBDA$ wird verwendet, um das Gewicht des L1-Verlustes im Generatorverlust zu steuern. Ein hoher Wert von $LAMBDA$ betont die Bedeutung der Inhaltsähnlichkeit zwischen den generierten und den Zielbildern. Die Anzahl der Trainingsepochen ist auf 450 gesetzt, was darauf hindeutet, dass das Modell eine umfangreiche Trainingsdauer durchläuft, um eine optimale Leistung zu erreichen.
\section{CycleGAN: Ergebnisse und objekte Bewertung}
\section{Vergleich und Bewertung von Pix2Pix und CycleGAN}
\section{Vergleich von Pix2Pix und CycleGAN}
- Matching Paare von Bildern sind ebenfalls für das Training nicht nötig (crewall)
- Macht die Datenvorbereitung einfacher und öffnet neue Techniken für Applikationen (crewall)