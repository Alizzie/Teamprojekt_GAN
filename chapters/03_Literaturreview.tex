\chapter{Literaturreview}
GANs haben sich als bedeutende Methode in der Bildgenerierung etabliert. Besonders zwei Varianten, Pix2Pix und CycleGAN, verdienen besondere Aufmerksamkeit.

Pix2Pix, eingeführt von Isola et al. (2017), fokussiert sich auf die direkte Zuordnung von Eingabe- zu Ausgabe-Bildern. Dies macht es besonders effektiv für Bild-zu-Bild-Übersetzungen, wie die Umwandlung von Graustufenbildern in Farbbilder. Der eingesetzte PatchGAN-Diskriminator ermöglicht eine präzise Unterscheidung zwischen generierten und realen Bildern.

CycleGAN, vorgestellt von Zhu et al. (2017), ermöglicht die Bildübersetzung zwischen zwei Domänen ohne gepaarte Trainingsdaten. Der Diskriminator von CycleGAN betrachtet das gesamte Bild, was eine umfassende Beurteilung der Unterschiede zwischen realen und generierten Bildern ermöglicht.

Die Forschung zu GANs hat bedeutende Fortschritte gemacht, und die Anwendung von Pix2Pix und CycleGAN hat sich als besonders wirkungsvoll erwiesen. In den folgenden Abschnitten werden Anwendungen, Architekturen und Herausforderungen dieser Modelle genauer beleuchtet.