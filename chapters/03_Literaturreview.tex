\chapter{Literaturreview}

Die Entstehung und Evolution der Generative Adversarial Networks (GANS) stehen für einen Wendepunkt in der Geschichte der Bildverarbeitung und der künstlichen Intelligenz. Das grundlegende Konzept der GANs wurde 2014 von Ian Goodfellow und seinen Kollegen in ihrem Papier "Generative Adversarial Nets" vorgestellt \cite{Goodfellow2014}.\newline
Die GAN-Technologie entwickelte sich in den folgenden Jahren rasant weiter. Durch die kontinuierliche Forschung und Entwicklung neuer Methoden und Techniken wurden frühe Herausforderungen wie instabiles Training und das Phänomen des Mode-Kollapses, bei dem Modelle dazu neigen, eingeschränkte Vielfalt zu erzeugen, gelöst. Die Stabilisierung der Netzwerke und die Verbesserung der Trainingsprozesse wurden durch bedeutende Beiträge unterstützt, wie die Arbeit von Salimans et al. (2016) mit „Improved Techniques for Training GANs“ \cite{Salimans.2016}. \newline
Die Verwendung von GANs hat sich schnell über die reine Bildgenerierung hinaus ausgedehnt. Im Papier "Image-to-Image Translation with Conditional Adversarial Networks“ von Isola et al. (2017) stellten sie das Pix2Pix-Modell vor, das GANs für die Bild-zu-Bild-Übersetzung verwenden \cite{PhillipIsola.}. Ebenso erweiterte das von Zhu et al. (2017) vorgestellte CycleGAN-Modell die Möglichkeiten der Bild-zu-Bild-Übersetzung für unpaarige Datensätze, was die Flexibilität von GANs in der Anwendung weiter erhöhte \cite{Zhu.2017}. \newline
Jedes dieser Modelle bringt eine andere Architektur der Generatoren und Diskriminatoren hervor, bei dem verschiedene Herausforderungen angegangen werden. 