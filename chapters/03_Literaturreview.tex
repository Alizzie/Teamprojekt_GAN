\chapter{Literaturreview}

Die Entstehung und Evolution der Generative Adversarial Networks (GANS) markieren für einen Wendepunkt in der Geschichte der Bildverarbeitung und der künstlichen Intelligenz. Das grundlegende Konzept der GANs wurde 2014 von Ian Goodfellow und seinen Kollegen in ihrem Papier "Generative Adversarial Nets" vorgestellt \cite{IanJ.GoodfellowJeanPougetAbadieMehdiMirzaBingXuDavidWardeFarleySherjilOzairAaro.2014}.\newline
In den darauffolgenden Jahren entwickelte sich die GAN-Technologie rasant weiter, weit über die bloße Bildgenerierung hinaus. GANs werden nicht nur für die Generierung von Bildern, sondern auch für die Wiederherstellung von relevanten Informationen in Trainingsdatensätzen sowie für Objekterkennung, Segmentierung und Klassifizierung eingesetzt. Unterschiedliche GAN-Modelle wurden für verschiedene Anwendungsbereiche entwickelt, die jeweils unterschiedliche Stärken aufweisen \cite{Aggarwal.2021}.
\\\newline
Im Rahmen des Papiers 'Image-to-Image Translation with Conditional Adversarial Networks' von Isola et al. (2017) wurde somit das Pix2Pix-Modell vorgestellt, das GANs für die Bild-zu-Bild-Übersetzung nutzt \cite{PhillipIsola.}. Ebenso erweiterte das von Zhu et al. (2017) präsentierte CycleGAN-Modell die Möglichkeiten der Bild-zu-Bild-Übersetzung für unpaarige Datensätze, was die Flexibilität von GANs in der Anwendung weiter erhöhte \cite{Zhu.2017}. Weitere bedeutende Modelle, die in der Forschung Beachtung finden, sind unter anderem das Conditional GAN (CGAN) \cite{Mirza.2014}, Wasserstein GAN (WGAN) \cite{Arjovsky.2017} und Deep Convolutional GAN (DCGAN) \cite{Krizhevsky.2017}.
\\\newline
Trotz dieser Fortschritte bleibt das Training von GANs eine anspruchsvolle Aufgabe, die verschiedene Herausforderungen mit sich bringt. Um diesen zu begegnen, wurden verschiedene Methoden, Techniken und Architekturen vorgeschlagen. Schwierigkeiten wie instabiles Training und Mode-Kollaps, bei dem Modelle dazu tendieren, eingeschränkte Vielfalt zu erzeugen, werden in der Arbeit von Salimans et al. (2016) \cite{Salimans.2016} zu "Improved Techniques for Training GANs" adressiert, um die Stabilisierung der Netzwerke und die Optimierung des Trainingsprozesses zu unterstützen.
Eine weiterführende Studie \cite{Hong.2020} präsentiert diverse GAN-Architekturen und Methoden, um das Problem des Mode-Kollapses zu überwinden. Dabei werden die Architekturen und Methoden variiert, um die Einsatzmöglichkeiten zu erweitern \cite{Jain.2020, Eckerli.2021, Sharma.2022} und neue Aspekte der GAN-Technologie zu erforschen \cite{Srivastava.2017}.
\\\newline

Im Kontext potenzieller gesellschaftlicher Auswirkungen können GANs erhebliche Herausforderungen im Bereich der Cybersicherheit darstellen, insbesondere hinsichtlich Gesichtserkennung und Passwortentschlüsselung \cite{Hitaj.912017}. Aggarwal deutete in seiner Arbeit bereits an, dass GANs in Bezug auf die menschliche Bildsynthese Besorgnis erregen könnten \cite{Aggarwal.2021}. Diese Bedenken eröffnen neue Forschungsrichtungen für GANs, die darauf abzielen, ethische Standards in der Künstlichen Intelligenz zu wahren.
\\
Die fortlaufende Forschung in diesem Bereich strebt danach, die Leistungsfähigkeit von GANs zu verbessern und neue Perspektiven für deren Nutzung in der Bildverarbeitung und künstliche Intelligenz zu erschließen.