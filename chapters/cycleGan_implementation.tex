\section{Implementierung der CycleGAN-Architektur}
Nach den theoretischen Grundlagen der CycleGAN-Architektur und der Datenvorverarbeitung in den vorherigen Abschnitten, wird nun die konkrete Implementierung der Architekturen unter Verwendung von TensorFlow und Keras vorgestellt.

\subsection{Generator und Diskriminator}
Die Generator- und Diskriminatorarchitekturen wurden entsprechend der im Grundlagenkapitel erklärten theoretischen Grundlagen umgesetzt, insbesondere den Empfehlungen von Zhu et al. (2017)\cite{Zhu.2017} folgend. Der Generator setzt sich aus einem Enkoder-Block, sechs Residualblöcken und einem Dekoder-Block zusammen \ref{fig:cycleGanGeneratorArchitecture}.
Der Diskriminator wurde als sequentielles Modell implementiert und umfasst mehrere Convolutional Schichten, konzipiert als PatchGAN. 
Nach jeder Convolutional Schicht beim Generator und Diskriminator, außer bei der letzten, folgt dabei eine Instanznormalisierung und ReLU-Aktivierung.
Die spezifischen Implementierungsdetails, inklusive der Helferfunktionen, sind im beigefügten Code zu finden.

\subsection{Verlustfunktion}
Während des Trainings werden verschiedene Verlustfunktionen verwendet, um sicherzustellen, dass der Generator qualitativ hochwertige Bilder generiert und dass die Transformationen zwischen den Domänen konsistent sind. Die zentralen Verlustfunktionen, insbesondere die Gesamtverlustfunktionen für den Generator und den Diskriminator, werden im Folgenden erläutert. 
\\
Für die Klassifizierung, ob es sich um echte oder generierte Bilder handelt, wird in der Implementierung der \textit{BinaryCrossentropy}-Verlustfunktion aus TensorFlow/Keras verwendet. 
Dieser berechnet den binären Kreuzentropieverlust zwischen den Zielwerten und den Vorhersagen\footnote{\url{https://www.tensorflow.org/api_docs/python/tf/keras/losses/BinaryCrossentropy}}.

\begin{lstlisting}[language=pyhaff, caption={Initialisierung des BinaryCrossentropy-Verlustfunktion}, label={cod:binaryCrossentropy}]
loss_obj = tf.keras.losses.BinaryCrossentropy(from_logits=True)
\end{lstlisting}


\subsubsection{Gesamtverlust des Generators}
Der Gesamtverlust des Generators setzt sich aus dem adversariellen Verlust und dem Zykluskonsistenz-Verlust zusammen. Optional kann der Identitätsverlust berücksichtigt werden, was in der Implementierung erfolgte. Die Integration des Identitätsverlusts stellt sicher, dass das Modell die Struktur des Originalbildes beibehält, was zu einem konsistenteren Transformationsprozess führt. Die spezifischen Implementierungen der adversariellen Verlustfunktion des Generators, des Cycle Consistency Loss und des Identitätsverlusts sind im Code-Anhang zu finden.
\begin{lstlisting}[language=pyhaff, caption={Vorverarbeitung des Datensatzes: Jittering}, label={cod:cycleGanGeneratorVerlust}]
def generator_loss(real_x, cycled_x, real_y, cycled_y, identity, discriminator_generated, step):
    gan_loss = generator_adversarial_loss(discriminator_generated)
    cycle_loss = cycle_consistency_loss(real_x, cycled_x) + cycle_consistency_loss(real_y, cycled_y)
    id_loss = identity_loss(real_x, identity)
    total_loss = gan_loss + cycle_loss + id_loss
    return total_loss
\end{lstlisting}

\subsubsection{Gesamtverlust des Diskriminators}
Die Gesamtverlustfunktion des Diskriminators setzt sich aus dem adversariellen Verlust für echte und generierte Bilder zusammen. Der Diskriminator wird entsprechend trainiert, echte Bilder als \"1\" und generierte Beispiele als \"0\" zu klassifizieren. Die Multiplikation des Gesamtverlustes mit dem Wert 0.5 bildet den Durchschnitt des Gesamtverlustes und stellt sicher, dass die Gradientenaktualisierung während des Prozesses angemessen skaliert wird.

\begin{lstlisting}[language=pyhaff, caption={Vorverarbeitung des Datensatzes: Jittering}, label={cod:cycleGanDiscriminatorVerlust}]
def discriminator_adversarial_loss(real, generated):
    real_loss = loss_obj(tf.ones_like(real), real)
    generated_loss = loss_obj(tf.zeros_like(generated), generated)
    total_disc_loss = real_loss + generated_loss
    return total_disc_loss * 0.5
\end{lstlisting}


\lstinputlisting[language=pyhaff, caption=CycleGAN Generator in Tensorflow]{code/CycleGan_Generator.txt}
\lstinputlisting[language=pyhaff, caption=CycleGAN Diskriminator in Tensorflow]{code/CycleGan_Diskriminator.txt}
