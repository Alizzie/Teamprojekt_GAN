\section{Training- und Testdaten}
Das Training und die Evaluierung von Generative Adversarial Networks (GANs) erfordern die klare Definition von Trainings- und Testdatensätzen. Der entscheidende Unterschied zwischen diesen Datensätzen besteht darin, dass das Modell während des Trainings auf den Trainingsdaten optimiert wird, während die Testdaten verwendet werden, um die Leistung und die Generalisierungsfähigkeiten des Modells zu bewerten.

\subsection{Datenladung für GAN-Training}
Für das effektive Training von GANs ist der Zugriff auf qualitativ hochwertige Datensätze von entscheidender Bedeutung. In diesem Kontext bietet TensorFlow eine umfassende Sammlung öffentlich verfügbarer Datensätze. Die verwendeten Datensätze werden von der Quelle \url{https://efrosgans.eecs.berkeley.edu} heruntergeladen und lokal extrahiert.

\lstinputlisting[language=pyhaff, caption=Laden eines Datensatzes von einer URL]{code/LadenDatensatz.txt}

Die Transformation und Vorverarbeitung der Bilddaten erfolgt durch die TensorFlow-Datensatz-API. Diese API bietet eine effiziente Datenpipeline für das Laden und Verarbeiten von Daten, insbesondere für den Einsatz in Machine-Learning-Modellen. In den folgenden Implementierungen wird ein Dataset durch eine Liste von Dateipfaden als Zeichenketten erzeugt.

\begin{lstlisting}[language=pyhaff, caption={Erzeugung eines Tensorflow-Dataset aus CycleGAN Implementierung}, label={cod:createDataset}]
train_horses =  tf.data.Dataset.list_files(str(PATH / 'trainA/*.jpg')) 
\end{lstlisting}

\subsection{Vorverarbeitung des Datensatzes}
Um die Leistung der GAN-Modelle zu verbessern, werden vor dem Training Variationen in den Trainingsdaten eingeführt. Dieser Prozess, der Datenjittering und Normalisierung umfasst, trägt dazu bei, das Modell robuster zu machen und eine bessere Konvergenz während des Trainings zu erreichen.
\newpage
\begin{lstlisting}[language=pyhaff, caption={Vorverarbeitung des Datensatzes: Normalisierung}, label={cod:normalizing}]
def normalize(image):
    image = (image / 127.5) - 1
    return image
\end{lstlisting}

Datenjittering wird durchgeführt, indem die heruntergeladenen Bilder (256x256) auf eine größere Größe (286x286) mit der Nearest-Neighbor-Methode skaliert. Bei dieser Methode wird für jedes Pixel im skalierten Bild der Farbwert des nächstgelegenen Pixels im Originalbild übernommen, um eine einfache und effiziente Skalierung zu ermöglichen. Das Bilder werden daraufhin zufällig zugeschnitten. Darüber hinaus wird eine zufällige Spiegelung angewendet.

\begin{lstlisting}[language=pyhaff, caption={Vorverarbeitung des Datensatzes: Jittering}, label={cod:jittering}]
def random_crop(image):
    cropped_image = tf.image.random_crop(image, size=(IMG_HEIGHT, IMG_WIDTH, 3))
    return cropped_image

def random_jitter(image):
    image = tf.image.resize(image, [286, 286], method=tf.image.ResizeMethod.NEAREST_NEIGHBOR)
    image = random_crop(image)
    image = tf.image.random_flip_left_right(image)
    return image
\end{lstlisting}

Zusätzlich dazu werden die Bildpfade geladen und in das resultierende JPEG-Format decodiert.

\begin{lstlisting}[language=pyhaff, caption={Vorverarbeitung des Datensatzes: Jittering}, label={cod:jittering}]
def load_image(image_path):
    image = tf.io.read_file(image_path)
    image = tf.io.decode_jpeg(image, channels=3)
    image = tf.cast(image, tf.float32)
    return image
\end{lstlisting}

Die vorverarbeiteten Bilder werden dann in TensorFlow-Datasets integriert. Nachfolgend wird der Trainingsdatensatz zufällig gemischt und in Batches gruppiert, wodurch sichergestellt wird, dass das Modell nicht von der Reihenfolge der Datenpunkte beeinflusst wird.
\newpage 
\lstinputlisting[language=pyhaff, caption=Integration der vorverarbeiteten Bilder in Tensorflow-Datasets]{code/LadenBild.txt}

Diese umfassende Vorverarbeitung stellt sicher, dass das GAN-Modell auf optimal vorbereiteten Daten trainiert wird, um eine maximale Leistung und Generalisierungsfähigkeit zu erreichen.


\section{Implementierung der CycleGAN-Architektur}
Nach den theoretischen Grundlagen der CycleGAN-Architektur und der Datenvorverarbeitung in den vorherigen Abschnitten, wird nun die konkrete Implementierung der Architekturen unter Verwendung von TensorFlow und Keras vorgestellt 
\\
Die Generator- und Diskriminatorarchitekturen wurden entsprechend der im Grundlagenkapitel erklärten theoretischen Grundlagen umgesetzt, insbesondere den Empfehlungen von Zhu et al. (2017)\cite{Zhu.2017} folgend. Der Generator setzt sich aus einem Enkoder-Block, mehreren Residualblöcken und einem Dekoder-Block zusammen. Der Diskriminator wurde als sequentielles Modell implementiert und umfasst mehrere Convolutional-Schichten, konzipiert als PatchGAN.

Die spezifischen Implementierungsdetails sind im beigefügten Code zu finden.

\subsection{Verlustfunktion}
Während des Trainings werden verschiedene Verlustfunktionen verwendet, um sicherzustellen, dass der Generator qualitativ hochwertige Bilder generiert und dass die Transformationen zwischen den Domänen konsistent sind. Die zentralen Verlustfunktionen, insbesondere die Gesamtverlustfunktionen für den Generator und den Diskriminator, werden im Folgenden erläutert. 
\\
Für die Klassifizierung, ob es sich um echte oder generierte Bilder handelt, wird in der Implementierung der \textit{BinaryCrossentropy}-Verlustfunktion aus TensorFlow/Keras verwendet. Dieser  berechnet den binären Kreuzentropieverlust zwischen den Zielwerten und den Vorhersagen \footnote{\url{https://www.tensorflow.org/api_docs/python/tf/keras/losses/BinaryCrossentropy}}.

\begin{lstlisting}[language=pyhaff, caption={Vorverarbeitung des Datensatzes: Jittering}, label={cod:jittering}]
loss_obj = tf.keras.losses.BinaryCrossentropy(from_logits=True)
\end{lstlisting}


\subsubsection{Gesamtverlust des Generators}
Der Gesamtverlust des Generators setzt sich aus dem adversariellen Verlust und dem Zykluskonsistenz-Verlust zusammen. Optional kann der Identitätsverlust berücksichtigt werden, was in der Implementierung erfolgte. Die Integration des Identitätsverlusts stellt sicher, dass das Modell die Struktur des Originalbildes beibehält, was zu einem konsistenteren Transformationsprozess führt. Die spezifischen Implementierungen der adversariellen Verlustfunktion des Generators, des Cycle Consistency Loss und des Identitätsverlusts sind im Code-Anhang zu finden.
\begin{lstlisting}[language=pyhaff, caption={Vorverarbeitung des Datensatzes: Jittering}, label={cod:jittering}]
def generator_loss(real_x, cycled_x, real_y, cycled_y, identity, discriminator_generated, step):
    gan_loss = generator_adversarial_loss(discriminator_generated)
    cycle_loss = cycle_consistency_loss(real_x, cycled_x) + cycle_consistency_loss(real_y, cycled_y)
    id_loss = identity_loss(real_x, identity)
    total_loss = gan_loss + cycle_loss + id_loss
    return total_loss
\end{lstlisting}

\subsubsection{Gesamtverlust des Diskriminators}
Die Gesamtverlustfunktion des Diskriminators setzt sich aus dem adversariellen Verlust für echte und generierte Bilder zusammen. Der Diskriminator wird entsprechend trainiert, echte Bilder als "1" und generierte Beispiele als "0" zu klassifizieren. Die Multiplikation des Gesamtverlustes mit dem Wert 0.5 bildet den Durchschnitt des Gesamtverlustes und stellt sicher, dass die Gradientenaktualisierung während des Prozesses angemessen skaliert wird.

\begin{lstlisting}[language=pyhaff, caption={Vorverarbeitung des Datensatzes: Jittering}, label={cod:cycleGanDiscriminator}]
def discriminator_adversarial_loss(real, generated):
    real_loss = loss_obj(tf.ones_like(real), real)
    generated_loss = loss_obj(tf.zeros_like(generated), generated)
    total_disc_loss = real_loss + generated_loss
    return total_disc_loss * 0.5
\end{lstlisting}

\subsection{Training und Hyperparameter}
Das Training wurde über 300 Epochen durchgeführt, wobei die Modelle durch spezifische Optimierer optimiert wurden. In jedem Durchlauf wurde ein Bild aus den Trainingsdaten ausgewählt, und der Generator wurde auf dieses angewendet, um inkrementell die Verbesserung zu verfolgen. Die generierten Bilder wurden mittels \textit{plt.savefig()} aus der Matplotlib-Bibliothek gespeichert, um auch nach dem Training darauf zugreifen zu können.\\
Diese Vorgehensweise ermöglicht eine systematische Überprüfung der Modellleistung und eine visuelle Darstellung der Fortschritte während des Trainingsprozesses.
\\
Die Wahl der Hyperparameter, einschließlich Lernraten und des Lambda-Werts für den Cycle Consistency Loss, wurde durch abgestimmte Experimente ermittelt. Die Optimierer wurden sorgfältig ausgewählt, um eine effiziente Konvergenz während des Trainings zu gewährleisten.


\lstinputlisting[language=pyhaff, caption=CycleGAN Generator in Tensorflow]{code/CycleGan_Generator.txt}
\lstinputlisting[language=pyhaff, caption=CycleGAN Diskriminator in Tensorflow]{code/CycleGan_Diskriminator.txt}
