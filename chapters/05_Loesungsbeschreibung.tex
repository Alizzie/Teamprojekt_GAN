\chapter{Lösungsbeschreibung}
\section{Training- und Testdaten}
Das Training und die Evaluierung von Generative Adversarial Networks (GANs) erfordern die klare Definition von Trainings- und Testdatensätzen. Der entscheidende Unterschied zwischen diesen Datensätzen besteht darin, dass das Modell während des Trainings auf den Trainingsdaten optimiert wird, während die Testdaten verwendet werden, um die Leistung und die Generalisierungsfähigkeiten des Modells zu bewerten.

\subsection{Datenladung für GAN-Training}
Für das effektive Training von GANs ist der Zugriff auf qualitativ hochwertige Datensätze von entscheidender Bedeutung. In diesem Kontext bietet TensorFlow eine umfassende Sammlung öffentlich verfügbarer Datensätze. Die verwendeten Datensätze werden von der Quelle \url{https://efrosgans.eecs.berkeley.edu} heruntergeladen und lokal extrahiert.

\lstinputlisting[language=pyhaff, caption=Laden eines Datensatzes von einer URL]{code/LadenDatensatz.txt}

Die Transformation und Vorverarbeitung der Bilddaten erfolgt durch die TensorFlow-Datensatz-API. Diese API bietet eine effiziente Datenpipeline für das Laden und Verarbeiten von Daten, insbesondere für den Einsatz in Machine-Learning-Modellen. In den folgenden Implementierungen wird ein Dataset durch eine Liste von Dateipfaden als Zeichenketten erzeugt.

\begin{lstlisting}[language=pyhaff, caption={Erzeugung eines Tensorflow-Dataset aus CycleGAN Implementierung}, label={cod:createDataset}]
train_horses =  tf.data.Dataset.list_files(str(PATH / 'trainA/*.jpg')) 
\end{lstlisting}

\subsection{Vorverarbeitung des Datensatzes}
Um die Leistung der GAN-Modelle zu verbessern, werden vor dem Training Variationen in den Trainingsdaten eingeführt. Dieser Prozess, der Datenjittering und Normalisierung umfasst, trägt dazu bei, das Modell robuster zu machen und eine bessere Konvergenz während des Trainings zu erreichen.

//TODO Normalisierung auf [-1:1], begründung

\newpage
\begin{lstlisting}[language=pyhaff, caption={Vorverarbeitung des Datensatzes: Normalisierung}, label={cod:normalizing}]
def normalize(image):
    image = (image / 127.5) - 1
    return image
\end{lstlisting}

Datenjittering wird durchgeführt, indem die heruntergeladenen Bilder (256x256) auf eine größere Größe (286x286) mit der Nearest-Neighbor-Methode skaliert. Bei dieser Methode wird für jedes Pixel im skalierten Bild der Farbwert des nächstgelegenen Pixels im Originalbild übernommen, um eine einfache und effiziente Skalierung zu ermöglichen. Das Bilder werden daraufhin zufällig zugeschnitten. Darüber hinaus wird eine zufällige Spiegelung angewendet.

\begin{lstlisting}[language=pyhaff, caption={Vorverarbeitung des Datensatzes: Jittering}, label={cod:jittering}]
def random_crop(image):
    cropped_image = tf.image.random_crop(image, size=(IMG_HEIGHT, IMG_WIDTH, 3))
    return cropped_image

def random_jitter(image):
    image = tf.image.resize(image, [286, 286], method=tf.image.ResizeMethod.NEAREST_NEIGHBOR)
    image = random_crop(image)
    image = tf.image.random_flip_left_right(image)
    return image
\end{lstlisting}

Zusätzlich dazu werden die Bildpfade geladen und in das resultierende JPEG-Format decodiert.

\begin{lstlisting}[language=pyhaff, caption={Lesen eines Bildes}, label={cod:imageLoading}]
def load_image(image_path):
    image = tf.io.read_file(image_path)
    image = tf.io.decode_jpeg(image, channels=3)
    image = tf.cast(image, tf.float32)
    return image
\end{lstlisting}

Die vorverarbeiteten Bilder werden dann in TensorFlow-Datasets integriert. Nachfolgend wird der Trainingsdatensatz zufällig gemischt und in Batches gruppiert, wodurch sichergestellt wird, dass das Modell nicht von der Reihenfolge der Datenpunkte beeinflusst wird.
\newpage 
\lstinputlisting[language=pyhaff, caption=Integration der vorverarbeiteten Bilder in Tensorflow-Datasets]{code/LadenBild.txt}

Diese umfassende Vorverarbeitung stellt sicher, dass das GAN-Modell auf optimal vorbereiteten Daten trainiert wird, um eine maximale Leistung und Generalisierungsfähigkeit zu erreichen.


\section{Implementierung der Pix2PixGAN-Architektur}
\subsection{Generator}

Die Struktur des Generator in der Pix2Pix-Implementierung ist ein wesentlicher Aspekt, der die Leistungsfähigkeit des Modells bestimmt. Der Generator ist als U-Net-Architektur aufgebaut, die aus dem Encoder und Decoder bestehen die wiederum  aufeinanderfolgende Downsampling- und Upsampling-Schritten beinhalten.
\newline
Im Encoder-Teil des Generators wird das Downsampling durch eine Reihe von Convolutional Neuronal Network (CNN) Schichten realisiert, die durch die downsample-Funktion innerhalb des downstack definiert sind. 
\newpage
\begin{lstlisting}[language=pyhaff, caption={Downsampling-Schritt in Pix2Pix}, label={cod:Pix2PixGAN Generator}]
down_stack = [
	downsample(64, 4, apply_batchnorm=False),  # (batch_size, 128, 128, 64)
	downsample(128, 4),  # (batch_size, 64, 64, 128)
	downsample(256, 4),  # (batch_size, 32, 32, 256)
	downsample(512, 4),  # (batch_size, 16, 16, 512)
	downsample(512, 4),  # (batch_size, 8, 8, 512)
	downsample(512, 4),  # (batch_size, 4, 4, 512)
	downsample(512, 4),  # (batch_size, 2, 2, 512)
	downsample(512, 4),  # (batch_size, 1, 1, 512)
]
\end{lstlisting}

Die downsample-Funktion erstellt eine Downsampling-Schicht, die mittels einer Conv2D-Schicht mit spezifischen Filtern und Kernel-Größen die räumlichen Dimensionen der Eingabebilder reduziert. Zur Verbesserung der Stabilität und Leistungsfähigkeit des Modells integriert die Funktion optional eine Batch-Normalisierung. Diese Normalisierung reguliert und standardisiert die Ausgabe der Conv2D-Schicht, was dazu beiträgt, das Training effizienter zu gestalten.
Darüber hinaus beinhaltet die Funktion eine LeakyReLU-Aktivierung, eine Variation der herkömmlichen ReLU-Aktivierungsfunktion. LeakyReLU ermöglicht es, dass auch für negative Eingabewerte ein kleiner Gradient erhalten bleibt, wodurch das Problem der inaktiven Neuronen, bekannt als "sterbende ReLUs", vermieden wird.

\begin{lstlisting}[language=pyhaff, caption={Downsampling-Schicht in Pix2Pix}, label={cod:Pix2PixGAN Generator}]
def downsample(filters, size, apply_batchnorm=True):
	initializer = tf.random_normal_initializer(0., 0.02)
	result = tf.keras.Sequential()
	result.add(
	tf.keras.layers.Conv2D(filters, size, strides=2, padding='same', kernel_initializer=initializer, use_bias=False))
	
	if apply_batchnorm:
		result.add(tf.keras.layers.BatchNormalization())
	
	result.add(tf.keras.layers.LeakyReLU())
	return result
\end{lstlisting}

Im Anschluss daran erfolgt das Upsampling im Decoder-Teil des Generators, das durch die upsample-Funktion innerhalb des upstack repräsentiert wird. Diese Schichten arbeiten daran, die Merkmale auf ein höher ausgelöstes Format zu projizieren und die Bildgröße wiederherzustellen.
\newpage
\begin{lstlisting}[language=pyhaff, caption={Upsampling-Schritt in Pix2Pix}, label={cod:Pix2PixGAN Generator}]
up_stack = [
	upsample(512, 4, apply_dropout=True),  # (batch_size, 2, 2, 1024)
	upsample(512, 4, apply_dropout=True),  # (batch_size, 4, 4, 1024)
	upsample(512, 4, apply_dropout=True),  # (batch_size, 8, 8, 1024)
	upsample(512, 4),  # (batch_size, 16, 16, 1024)
	upsample(256, 4),  # (batch_size, 32, 32, 512)
	upsample(128, 4),  # (batch_size, 64, 64, 256)
	upsample(64, 4),  # (batch_size, 128, 128, 128)
]
\end{lstlisting}

Die upsampling-Funktion verwendet eine spezielle Art von Convolutional Layer, die Conv2DTranspose-Schicht, um die Bildgröße zu erhöhen. Diese Schicht kehrt den Prozess einer Convolutional Layer Schicht um, indem sie die Eingabedaten expandiert, was für das Wiederherstellen einer größerem Bildgröße im Generator unerlässlich ist. Zusätzlich zur Conv2DTranspose-Schicht integriert die upsample-Funktion eine Batch-Normalisierung, die zur Stabilisierung des Lernprozesses beiträgt, indem sie die Ausgaben der Conv2DTranspose-Schicht normalisiert. 

\begin{lstlisting}[language=pyhaff, caption={Upsampling-Schritt in Pix2Pix}, label={cod:Pix2PixGAN Generator}]
def upsample(filters, size, apply_dropout=True):
	initializer = tf.random_normal_initializer(0., 0.02)
	result = tf.keras.Sequential()
	result.add(
	tf.keras.layers.Conv2DTranspose(filters, size, strides=2, padding='same', kernel_initializer=initializer, 
	use_bias=False))
	result.add(tf.keras.layers.BatchNormalization())
	
	if apply_dropout:
		result.add(tf.keras.layers.Dropout(0.5))
	
	result.add(tf.keras.layers.ReLU())
	return result
\end{lstlisting} 
\newpage
Dies ist ein wichtiger Schritt, um die interne Kovariantenverschiebung zu reduzieren und die Leistung des Modells zu verbessern. Ein weiteres wichtiges Merkmal der Funktion ist die optional Anwendung von Dropout um Overfitting zu vermeiden. Dies trägt dazu bei dass das Modell robustere und generalisierbare Merkmale lernt. Schließlich wird eine ReLU-Aktivierungsfunktion angewendet, die dafür sorgt, dass das Modell nicht-lineare Zusammenhänge lernt. 
\\\newline
Die Skip-Verbindungen werden im Generator durch die Speicherung und spätere Verwendung der Ausgaben der Downsampling-Schichten in der skips-Liste realisiert. Nach dem Downsampling-Prozess werden diese gespeicherten Ausgaben in umgekehrter Reihenfolge durchlaufen und mit den Ausgaben der Upsampling-Schichten mittels einer Concatenate-Operation verbunden. Diese Kombination von hoch- und niedrigstufigen Merkmalen führt zu einer detaillierteren und genaueren Bildrekonstruktion.
\newline
Schließlich wird das endgültige Bild durch die letzte Schicht des Generators erzeugt, eine Conv2DTranspose-Schicht, die die Ausgabe des Generators darstellt. Diese letzte Schicht spielt eine entscheidende Rolle bei der Erzeugung des endgültigen Bildes, das die kombinierten Merkmale aus dem gesamten Netzwerk nutzt.

\begin{lstlisting}[language=pyhaff, caption={Skip Verbindungen in Pix2Pix}, label={cod:Pix2PixGAN Generator}]
	initializer = tf.random_normal_initializer(0., 0.02)
	last = tf.keras.layers.Conv2DTranspose(OUTPUT_CHANNELS, 4, strides=2, padding='same',
	kernel_initializer=initializer, activation='tanh')  # (batch_size, 256, 256, 3)
	x = inputs
	
	# Downsampling through the model
	skips = []
	for down in down_stack:
		x = down(x)
		skips.append(x)
	skips = reversed(skips[:-1])
	
	# Upsampling and establishing the skip connections
	for up, skip in zip(up_stack, skips):
		x = up(x)
		x = tf.keras.layers.Concatenate()([x, skip])
	x = last(x)
	
	return tf.keras.Model(inputs=inputs, outputs=x)
\end{lstlisting}

\subsection{Diskriminator}
Der Diskriminator startet mit der Initialisierung seiner Eingabeschichten. Er empfängt zwei seperate Bilder - ein Eingabebild ($inp$) und ein Zielbild ($tar$). Diese Bilder werden dann entlang ihrer Farbkanäle zu einem einzigen Bild zusammengefügt.\newline
Im Kern des Diskriminators des Pix2Pix-Modells finden sich mehrere Downsampling- oder Convolutional Layer, die eine Schlüsselrolle bei der Bewertung des Eingabebildes spielen. jede dieser Schichten führt Konvolutionen durch, um die Merkmale und Texturen aus den Bildern zu extrahieren. Dabei arbeitet jeder Convolutional Layer mit einem kleinen Bereich des Eingabebildes. Dieser Bereich gleitet über das gesamte Bild und bewertet bei jedem Schritt einen kleinen Teil, bekannt als $"Patch"$. Die Größe dieses Bereiches und damit die Größe des bewerteten Patches, wird durch die Größe des Konvolutionskerns bestimmt. Diese konsequente Analyse von Patches ermöglicht es dem Diskriminator, die räumliche Auflösung des Bildes schrittweise zu reduzieren, was wiederum die Komplexität des Problems verringert und eine effektive Bewertung der lokalen Bildmerkmale ermöglicht. \newline
Nach den Downsampling-Schritten folgen Zero Padding und zusätzliche Convolutional Layer. Diese Schritte sind entscheidend, um den Diskriminator zu ermöglichen, feinere Details aus den Bildern herauszuarbeiten. \newline
Die letzte Schicht im Diskriminator ist ein Convolutional Layer, die eine Karte von Werten erzeugt. Jeder dieser Werte repräsentiert das Urteil des Diskriminators über einen bestimmten Patch des Bildes.\newline
Der Schlüssel des Diskriminators liegt im PatchGAN-Konzept. Dieses Konzept geschieht implizit durch die Art und Weise, wie die Convolutional Layer im Netzwerk strukturiert sind.
\newpage
\lstinputlisting[language=pyhaff, caption=Pix2Pix Diskriminator in Tensorflow]{code/Pix2Pix_Diskriminator.txt}

\newpage
\subsection{Verlustfunktion}
\subsubsection{Generatorverlust}
Der Generatorverlust im Pix2Pix-Modell besteht aus zwei wesentlichen Komponenten: dem adversariellen Verlust ($gan\_loss$) und dem L1-Verlust ($l1\_loss$). Der adversarielle Verlust wird durch Anwendung der $BinaryCrossentropy$-Funktion von TensorFlow ermittelt. Hierbei wird die Ausgabe des Diskriminators für generierte Bilder ($disc\_generated\_output$) mit einem Tensor, der ausschließlich aus Einsen besteht, verglichen. Dies dient dazu, die Effektivität des Generators bei der Erzeugung von Bildern zu bewerten, die für den Diskriminator von echten Bildern nicht unterscheidbar sind. Der L1-Verlust hingegen berechnet den mittleren absoluten Fehler zwischen dem vom Generator erzeugten Bild ($gen\_output$) und dem tatsächlichen Zielbild ($target$). Dieser Verlust trägt maßgeblich dazu bei, die inhaltliche Übereinstimmung und Ähnlichkeit des generierten Bildes mit dem Zielbild zu fördern. \newline
Der Gesamtverlust des Generators ($total\_gen\_loss$) ist die Summe des adversariellen Verlust und des L1-Verlusts, wobei der L1-Verlust mit einem Faktor $LAMBDA$ gewichtet wird. Die Gewichtung des L1-Verlusts hilft dabei, die strukturelle Integrität und die Genauigkeit des generierten Bildes zu verbessern,indem sie darauf abzielt, die Pixeldifferenzen zwischen dem generierten Bild und dem Zielbild zu minimieren.\newline
Nach der Berechnung des Gesamtverlusts werden Gradienten bezüglich der Generatorparameter berechnet ($gen\_tape.gradient$) und diese Gradienten werden dann verwendet, um den Generator mittels des Adam-Optimierers \\($generator\_optimizer.apply\_gradients$) zu aktualisieren. Dieser Prozess ist ein integraler Bestandteil des Trainings, da er dem Generator hilft, sich schrittweise zu verbessern und immer realistischere Bilder zu erzeugen.

\subsubsection{Diskriminatorverlust}
Der Diskriminatorverlust wird durch die $discriminator\_loss$-Funktion im Coe bestimmt. Diese Funktion enthält zwei Eingaben: $disc\_real\_output$, die Diskriminatorausgabe für das echte Bild und $disc\_generated\_output$, die Diskriminatorausgabe für das vom Generator erzeugte Bild. Der Verlust für echte Bilder ($real\_loss$) wird berechnet, indem die $BinaryCrossentropy$-Funktion zwischen $disc\_real\_output$ und einem Tensor aus Einsen angewendet wird. Dieser Schritt bewertet, wie gut der Diskriminator eichte Bilder als solche erkennen kann.
Der Verlust für generierte Bilder ($generated\_loss$) wird berechnet, indem die $BinaryCrossentropy$-Funktion zwischen $disc\_generated\_output$ und einem Tensor aus Nullen angewendet wird. Dies bewertet, wie gut der Diskriminator generierte Bilder als flasch erkennen kann.
\newline
Der Gesamtverlust des Diskriminators ($total\_disc\_loss$) ist die Summe von \\$real\_loss$ und $generated\_loss$. Diese Kombination zwingt den Diskriminator, zwischen echten und generierten Bildern besser zu unterscheiden.
\newline
Für die Optimierung des Diskriminator werden die Gradienten des Diskriminatorverlusts in Bezug auf die Diskriminatorparameter berechnet ($disc\_tape.gradient$). Diese Gradienten werden dann verwendet, um den Diskriminator mittels des Adam-Optimierers ($discriminator\_optimizer.apply\_gradients$) zu aktualisieren.




\section{Implementierung der CycleGAN-Architektur}
Nach den theoretischen Grundlagen der CycleGAN-Architektur und der Datenvorverarbeitung in den vorherigen Abschnitten, wird nun die konkrete Implementierung der Architekturen unter Verwendung von TensorFlow und Keras vorgestellt.

\subsection{Generator und Diskriminator}
Die Architekturen des Generators und Diskriminators wurden gemäß den theoretischen Grundlagen umgesetzt, insbesondere den Richtlinien von Zhu et al. (2017) \cite{Zhu.2017}. 

Der Generator besteht aus einem Encoder-Block, gefolgt von sechs Residual-\\Blöcken und einem Decoder-Block, welche in Abbildung \ref{fig:cycleGanGeneratorArchitecture} dargestellt ist.
Der Diskriminator wurde als sequentielles Modell implementiert und umfasst mehrere Convolutional Schichten, konzipiert als PatchGAN.

Nach jeder Convolutional Schicht im Generator und Diskriminator, abgesehen von der letzten, wird eine Instanznormalisierung und eine ReLU-Aktivierung angewendet. 
Die Instanznormalisierung dient dazu, die Aktivierungen zu normalisieren und das Training zu stabilisieren, indem sie die Eingaben jedes Minibatches normalisiert. Die ReLU-Aktivierung fördert die Einführung von Nichtlinearitäten in das Modell und ermöglicht es, komplexere Merkmale zu erfassen.

Für die Output-Schicht wird die tanh-Aktivierungsfuntkion angewendet. Diese Schicht begrenzt die Ausgabewerte auf den Bereich zwischen -1 und +1. Diese Begrenzung ist nicht nur wichtig, um eine konsistente Skalierung mit den Trainingsdaten sicherzustellen, sondern trägt auch zur Stabilisierung des Trainingsprozesses bei \cite{Radford.2015}.

Die spezifischen Implementierungsdetails, inklusive der Helferfunktionen, sind im beigefügten Code zu finden.

\newpage
\lstinputlisting[language=pyhaff, caption=CycleGAN Generator in Tensorflow]{code/CycleGan_Generator.txt}
\newpage
\lstinputlisting[language=pyhaff, caption=CycleGAN Diskriminator in Tensorflow]{code/CycleGan_Diskriminator.txt}


\newpage
\subsection{Verlustfunktion}
Während des Trainings werden verschiedene Verlustfunktionen verwendet, um sicherzustellen, dass der Generator qualitativ hochwertige Bilder generiert und dass die Transformationen zwischen den Domänen konsistent sind. Die zentralen Verlustfunktionen, insbesondere die Gesamtverlustfunktionen für den Generator und den Diskriminator, werden im Folgenden erläutert. 
\\
Für die Klassifizierung, ob es sich um echte oder generierte Bilder handelt, wird in der Implementierung der \textit{BinaryCrossentropy}-Verlustfunktion aus TensorFlow/Keras verwendet. 
Dieser berechnet den binären Kreuzentropieverlust zwischen den Zielwerten und den Vorhersagen\footnote{\url{https://www.tensorflow.org/api_docs/python/tf/keras/losses/BinaryCrossentropy}}.

\begin{lstlisting}[language=pyhaff, caption={Initialisierung des BinaryCrossentropy-\\Verlustfunktion}, label={cod:binaryCrossentropy}]
loss_obj = tf.keras.losses.BinaryCrossentropy(from_logits=True)
\end{lstlisting}


\subsubsection{Gesamtverlust des Generators}
Der Gesamtverlust des Generators setzt sich aus dem adversariellen Verlust und dem Zykluskonsistenz-Verlust zusammen. Optional kann der Identitätsverlust berücksichtigt werden, was zu einem konsistenteren Transformationsprozess führt (Code \ref{cod:cycleGanGeneratorVerlust}). Die Integration des Identitätsverlusts gewährleistet die Bewahrung der Struktur des Originalbildes.

Im adversariellen Verlust, implementiert durch die Funktion \\$generator\_adversarial\_loss$, werden die generierten Bilder mit einem Tensor aus Einsen verglichen, welcher die Zielwerte für ''echt'' repräsentiert. Die Verlustberechnung erfolgt mittels einer spezifizierten Verlustfunktion.

Für den Zykluskonsistenz-Verlust ($cycle\_loss$-Funktion) wird der mittlere absolute Unterschied zwischen einem echten Bild und seiner zyklisch transformierten Version berechnet. Dieser L1-Verlust stellt sicher, dass die Übersetzung zwischen den Domänen und zurück nahe an der Identitätsabbildung liegt.

Die Funktion $identity\_loss$ ermittelt den Identitätsverlust, wobei der L1-Verlust zwischen einem echten Bild und seiner übersetzten Version in derselben Domäne verwendet wird. 

Zur Berücksichtigung ihrer Bedeutung für den Gesamtverlust wird der Verlust an Zykluskonsistenz und Identität jeweils mit einem gewichteten Faktor multipliziert.

Der Code-Anhang enthält detaillierte Implementierungen der genannten Verlustfunktionen.
\newpage
\begin{lstlisting}[language=pyhaff, caption={Gesamtverlust des Generators in CycleGAN}, label={cod:cycleGanGeneratorVerlust}]
def generator_loss(real_x, real_y, cycled_x, cycled_y, disc_fake, identity):
    gen_loss = generator_adversarial_loss(disc_fake)
    total_cycle_loss = cycle_loss(real_x, cycled_x) + cycle_loss(real_y, cycled_y)
    id_loss = identity_loss(real_y, identity)
    total_gen_loss = gen_loss + total_cycle_loss + id_loss
    return total_gen_loss
\end{lstlisting}

\subsubsection{Gesamtverlust des Diskriminators}
Die Gesamtverlustfunktion des Diskriminators setzt sich aus dem adversariellen Verlust für echte und generierte Bilder zusammen. Der Diskriminator wird dementsprechend trainiert, echte Bilder als Einsen und generierte Beispiele als Nullen zu klassifizieren.

Zu Beginn wird der Verlust berechnet, wenn der Diskriminator echte Bilder betrachtet. Hierbei kommt die BinaryCrossentropy-Verlustfunktion zum Einsatz, die den Verlust zwischen den echten Vorhersagen ($real$) und den Zielwerten berechnet. Anschließend erfolgt die Berechnung des Verlusts zwischen den generierten Vorhersagen ($generated$) und den Zielwerten.

Der Gesamtverlust ergibt sich als Summe der beiden Teilverluste. Um sicherzustellen, dass die Gradientenaktualisierung während des Trainings angemessen skaliert wird, erfolgt eine Multiplikation des Gesamtverlustes mit dem Wert 0.5, was einer Bildung des Durchschnitts des Gesamtverlustes entspricht.

\begin{lstlisting}[language=pyhaff, caption={Gesamtverlust des Diskriminators in CycleGAN}, label={cod:cycleGanDiscriminatorVerlust}]
def discriminator_adversarial_loss(real, generated):
    real_loss = loss_obj(tf.ones_like(real), real)
    generated_loss = loss_obj(tf.zeros_like(generated), generated)
    total_disc_loss = real_loss + generated_loss
    return total_disc_loss * 0.5
\end{lstlisting}