\subsection{Pix2Pix-Kernkonzepte}
Die Bildverarbeitung hat in den letzten Jahren durch den Einsatz tiefer neuronaler Netzwerke erhebliche Fortschritte gemacht. Im Mittelpunkt vieler dieser Fortschritte steht die U-Net-Architektur, die, die speziell für die Bildsegmentierung entwickelt wurde. Diese Architektur zeichnet sich durch ihre angeklügelte Kombination aus Encoder- und Decoder- Strukturen sowie durch den Einsatz von Skip-Verbindungen aus. \newline
Bei der Encoder-Decoder-Struktur handelt es sich um einen Ansatz, bei dem das Eingangsbild zunächst durch den Encoder schrittweise reduziert wird. Dieser Prozess dient dazu, wesentliche Merkmale des Bildes zu erfassen. Anschließend wird das Bild durch den Decoder wiederhergestellt, indem die zuvor extrahierten Merkmale verwendet werden. Während dieser Prozesse besteht jedoch das Risiko des Informationsverlustes, insbesondere in den tieferen Schichten des Netzwerks.
Um dieses Problem zu adressieren, führt die U-Net-Architektur Skip-Verbindungen ein. Diese direkten Verbindungen zwischen korrespondierenden Schichten des Encoders und Decoders sorgen dafür, dass Detailinformationen nicht verloren gehen. Genauer gesagt, ermöglichen diese Verbindungen den direkten Informationsfluss zwischen jeweils äquivalenten Schichten, wodurch die Rekonstruktion des Bildes im Decoder mit einer höheren Genauigkeit erfolgt. \newline
Die Bedeutung von Skip-Verbindungen zeigt sich insbesondere in Anwendungen wie der Bild-zu-Bild-Übersetzung. Hier muss oft ein Bild mit niedriger Auflösung in ein Bild mit hoher Auflösung überführt werden, ohne dass Details verloren gehen. Die U-Net-Architektur, die angereichert mit diesen Verbindungen ist, ermöglicht daher eine feinere Rekonstruktion, die sowohl globale als auch lokale Informationen berücksichtigt. \newline
Somit kann die U-Net-Architektur durch ihre Kombination aus Encoder-Decoder-Struktur und Skip-Verbindungen ein effektives Werkzeug für die Bildsegemtierung darstellen. Ihre Fähigkeit, sowohl globale Muster als auch feine Details zu berücksichtigen, macht sie zu einer bevorzugten Wahl für viele Bildverarbeitungsaufgaben

