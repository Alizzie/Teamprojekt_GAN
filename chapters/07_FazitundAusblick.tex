\chapter{Fazit und Ausblick}
Hier wird ein Fazit und ein Ausblick gegeben.

\section{Fazit}
Das Teamprojekt $"$Analyse, Design und Implementierung von unterschiedlichen Generative Adversarial Network Architekturen im Bereich der Bildverarbeitung$"$ bietet einen umfassenden Einblick in die Welt der Generative Adversarial Networks (GANs) und ihre Anwendungen in der Bildverarbeitung. Bereits in der Einleitung werden verschiedene Aspekte von GANs und ihre Relevanz in der Bildverarbeitung angesprochen. Im weiteren Verlauf der Arbeit wurden diese Aspekte vertieft und konkretisiert, wobei ein solides Fundament aus Fachtermini und Grundlagen im Bereich GANs gelegt wurde. Ein besonderes Augenmerk lag dabei auf den spezialisierten Frameworks Pix2Pix und CycleGAN.

Nicht nur theoretische Konzepte wurden beleuchtet, sondern auch eine praktische Umsetzung erfolgte. Die Modelle Pix2Pix und CycleGAN wurden nicht nur theoretisch diskutiert, sondern auch konkret implementiert und auf einen spezifischen Datensatz trainiert. Die Implementierung und das Training dieser Modelle erfolgten unter Einsatz der Frameworks TensorFlow und Keras, deren Flexibilität und Benutzerfreundlichkeit für die Entwicklung und Optimierung dieser komplexen Modelle entscheidend waren. Durch diesen praktischen Ansatz konnten die theoretischen Konzepte veranschaulicht und die Leistungsfähigkeit sowie die Herausforderungen der verschiedenen GAN-Architekturen in der realen Anwendung demonstriert werden. Die Ergebnisse des Trainingsprozesses bieten wertvolle Einblicke in die Möglichkeiten und Grenzen dieser innovativen Technologien im Bereich der Bildverarbeitung.