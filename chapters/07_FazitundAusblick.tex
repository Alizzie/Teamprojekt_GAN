\chapter{Fazit und Ausblick}
Hier wird ein Fazit und ein Ausblick gegeben.

\section{Fazit}
Das Teamprojekt $"$Analyse, Design und Implementierung von unterschiedlichen Generative Adversarial Network Architekturen im Bereich der Bildverarbeitung$"$ bietet einen umfassenden Einblick in die Welt der Generative Adversarial Networks (GANs) und ihre Anwendungen in der Bildverarbeitung. Bereits in der Einleitung werden verschiedene Aspekte von GANs und ihre Relevanz in der Bildverarbeitung angesprochen. Im weiteren Verlauf der Arbeit wurden diese Aspekte vertieft und konkretisiert, wobei ein solides Fundament aus Fachtermini und Grundlagen im Bereich GANs gelegt wurde. Ein besonderes Augenmerk lag dabei auf den spezialisierten Frameworks Pix2Pix und CycleGAN.

Nicht nur theoretische Konzepte wurden beleuchtet, sondern auch eine praktische Umsetzung erfolgte. Die Modelle Pix2Pix und CycleGAN wurden nicht nur theoretisch diskutiert, sondern auch konkret implementiert und auf einen spezifischen Datensatz trainiert. Die Implementierung und das Training dieser Modelle erfolgten unter Einsatz der Frameworks TensorFlow und Keras, deren Flexibilität und Benutzerfreundlichkeit für die Entwicklung und Optimierung dieser komplexen Modelle entscheidend waren. Durch diesen praktischen Ansatz konnten die theoretischen Konzepte veranschaulicht und die Leistungsfähigkeit sowie die Herausforderungen der verschiedenen GAN-Architekturen in der realen Anwendung demonstriert werden. Die Ergebnisse des Trainingsprozesses bieten wertvolle Einblicke in die Möglichkeiten und Grenzen dieser innovativen Technologien im Bereich der Bildverarbeitung.

Dir Ergebnisse zeigen, dass beide Modelle effektiv in der Bild-zu-Bild-Übersetzung sind, jedoch mit unterschiedlichen Stärken und Herausforderungen in Bezug auf die Art der Daten und die erzielte Bildqualität.\newline
Pix2Pix, das gepaarte Trainingsdaten benötigt, zeigte sich besonders effizient bei Aufgaben, bei denen eine direkte Korrespondenz zwischen Eingabe- und Ausgabebildern besteht. Dies wurde durch seine hohe Diskriminatorgenaugkeit unterstrichen, was auf eine effektive Unterscheidung zwischen echten und generierten Bildern hindeutet. Allerdings wurde auch festgestellt, dass eine zu hohe Genauigkeit des Diskriminators die Entwicklungsfähigkeit des Generators einschränken könnte.\newline

Im Gegensatz dazu ist CycleGAN für ungepaarte Daten konzipiert und demonstriert seine Stärke in der Zykluskonsistenz. Dies ist insbesondere bei Anwendungen nützlich, bei denen keine gepaarten Beispiele verfügbar sind. Die Ergebnisse zeigen, dass CycleGAN effektiv bei der Erhaltung des ursprünglichen Inhalts in der Zyklusübersetzung ist, was durch die exponentielle Steigerung des SSIM-Scores belegt wird.

Die Analyse der SSIM-Scores und Diskriminatorgenauigkeiten in beiden Modellen gab wertvolle Einblicke in ihre Fähigkeit, qualitativ hochwertige Bilder zu erzeugen. Bei Pix2Pix wurde beispielsweise eine deutliche Verbesserung der Bildqualität nach der Anpassung des Lambda-Wertes festgestellt. Bei CycleGAN hingegen wurde eine kontinuierliche Steigerung der Bildqualität im Laufe der Zeit beobachtet, besonders in Bezug auf die Zyklusübersetzung.

Die Auswirkungen von Trainingsparametern und der Trainingsdauer auf die Leistung der Modelle wurden ebenfalls gründlich untersucht. Insbesondere die langfristige Entwicklung der Generatoren von CycleGAN und die Anpassungen bei Pix2Pix zeigten, wie entscheidend die Wahl der richtigen Trainingsparameter ist. 