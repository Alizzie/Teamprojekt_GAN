\chapter{Fazit und Ausblick}
Hier wird ein Fazit und ein Ausblick gegeben.

\section{Fazit}
Im Zuge dieses Teamprojekts wurden zentrale Forschungsfragen im Bereich der Bildverarbeitung mit Generative Adversarial Networks (GANs) untersucht und beantwortet. Dabei standen insbesondere die Frameworks Pix2Pix und CycleGAN im Fokus, um die vielfältigen Anwendungsmöglichkeiten und Herausforderungen von GANs zu beleuchten. Durch die praktische Implementierung und das Training der Modelle Pix2Pix und CycleGAN auf einem spezifischen Datensatz konnten wir die Leistungsfähigkeit und Grenzen dieser Technologien demonstrieren. Die Ergebnisse zeigen, dass beide Modelle effektiv in der Bild-zu-Bild-Übersetzung sind, sich jedoch in ihren Stärken und Herausforderungen unterscheiden.

Während Pix2Pix für seine Effizienz bei gepaarten Trainingsdaten bekannt ist, demonstriert CycleGAN seine Stärke in der Zykluskonsistenz bei ungepaarten Daten. Die Analyse der SSIM-Scores und Diskriminatorgenauigkeiten und weiterer Metriken beider Modelle hat einen tiefgreifenden Einblick in ihre Fähigkeit zur Erzeugung qualitativ hochwertiger Bilder ermöglicht. Besonders auffällig war die kontinuierliche Verbesserung der Bildqualität bei CycleGAN im Laufe des Trainings sowie die Notwendigkeit der sorgfältigen Anpassung der Trainingsparameter bei Pix2Pix und CycleGAN. Diese Erkenntnisse unterstreichen die Bedeutung einer gezielten und durchdachten Konfiguration der Trainingsparameter für die erfolgreiche Anwendung von GANs in der Bildverarbeitung.

\section{Ausblick}
Der Ausblick für die weitere Forschung und Anwendung von Generative Adversarial Networks (GANs) im Bereich der Bildverarbeitung ist vielversprechend und bietet zahlreiche Möglichkeiten für Innovationen. Die Ergebnisse dieses Projekts legen nahe, dass zukünftige Arbeiten sich darauf konzentrieren könnten, die Effizienz und Genauigkeit von GAN-Modellen weiter zu verbessern, insbesondere in Bereichen, in denen hochpräzise Bild-zu-Bild-Übersetzungen erforderlich sind.

Des Weiteren könnte die Erforschung neuer Anwendungsbereiche, in denen GANs bisher noch nicht eingesetzt wurden, neue Horizonte eröffnen. Dies könnte die Entwicklung maßgeschneiderter Lösungen für spezifische Aufgabe oder kreative Prozesse beinhalten. Auch die Auswirkungen unterschiedlicher Datensätze und die Entwicklung von GANs, die effektiv mit weniger oder ungepaarten Daten arbeiten können, sind vielversprechende Forschungsfelder. Letztlich könnte die Erweiterung der ethischen und rechtlichen Rahmenbedingungen, insbesondere in Bezug auf die Verwendung generierter Bilder, eine Schlüsselrolle spielen, um das volle Potenzial von GANs in einer verantwortungsbewussten und nachhaltigen Weise zu nutzen.
