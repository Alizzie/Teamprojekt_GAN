\chapter*{Cycle GAN}

CycleGAN ist ein leistungsstarkes Modell im Bereich des maschinellen Lernens, das darauf abzielt, Bildübersetzungen zwischen verschiedenen Domänen ohne die Notwendigkeit für gepaarte Daten durchzuführen. Diese Technik wurde erstmals 2017 von Jun-Yan Zhu et al. in ihrer bahnbrechenden Arbeit "Unpaired Image-to-Image Translation using Cycle-Consistent Adversarial Networks" vorgestellt. Dieses Kapitel widmet sich der Funktionsweise von CycleGAN und seinen vielfältigen Anwendungen.

\subsection*{Funktionsweise von CycleGAN}

CycleGAN basiert auf dem GAN-Prinzip. Der Generator erzeut Bilder, während der Diskriminator zwischen den echten und generierten Bilder unterscheiden muss. Im Falle von CycleGAn wird das Modell erweitert, um die Übersetzung zwischen zwei Domänen zu ermöglichen.

\subsection*{Generator}

\subsection*{Diskriminator}

\subsection*{Cycle - Konsistenz}

\subsection*{Verlustfunktionen}
