\documentclass[12pt, a4paper, twoside, ngerman, openright, headsepline=true, footsepline=false]{scrbook}

% Präambel --> Nutzereingaben, Pakete und definierte Code-Darstellungen werden geladen.
% =======================================================================================
% Bitte die Angaben in den hintersten Klammern eintragen!
% Beispiel: \newcommand{\art}{Bachelorarbeit}

% Angaben für das Deckblatt
% ======================================================================================================
\newcommand{\art}{Master-Teamprojekt}
\newcommand{\thema}{Titel der Arbeit}
\newcommand{\themaEnglisch}{Titel of the Thesis} % Titel auf englisch
\newcommand{\autor}{Max Mustermann}
\newcommand{\matrikel}{420815}
\newcommand{\prof}{Prof. Dr. rer. nat. E.-G. Haffner}
\newcommand{\betrieb}{Bsp.-Unternehmen GmbH}    % bei Arbeit an der Hochschule weglassen
\newcommand{\betreuer}{Herr B. Treuer} %bei Arbeit an der Hochschule weglassen
\newcommand{\abgabedatum}{14. März 2022}


% Layout-Einstellungen
% ======================================================================================================
% Hier kann über die Angabe eines Zahlenwertes festgelegt werden, ob die Kapitel einen schwarzen Balken mit weißer Kapitelnummerierung am rechten Rand der rechten Seiten haben sollen oder nicht.
% Damit die Kapitelbalken angezeigt werden muss dem Zähler "kb" die Zahl 1 zugewiesen werden (\setcounter{kb}{1}). Damit die Kapitelbalken nicht erscheinen muss dem Zähler "kb" die Zahl 0 zugewiesen werden (\setcounter{kapitelbalken}{0}).
\newcounter{kb}
\setcounter{kb}{1}

% Hier kann eingestellt werden, ob der Anhang Kapitelbalken haben soll. 
\newcounter{akb}
\setcounter{akb}{1}
% Diese Datei "b_packages.tex" ist dafür gedacht alle Packages einzubinden, die benötigt werden. Außerdem werden einige Einstellungen vorgenommen.

%=== Einstellungen ==============================================
% Erstzeileneinzug ausschalten
\setlength{\parindent}{0cm}

%=== Text- und Sprachpackages ===================================
% Setzt die Inputcodierung zu utf8
\usepackage[utf8]{inputenc}
% korrekte Silbentrennung für Westeuropäische Sprachen
\usepackage[T1]{fontenc}

% neue deutsche Rechtschreibung, Anführungszeichen; als zweite Sprache wird Englisch verwendet
\usepackage[main=ngerman, english]{babel}

%=== Packages zum Layout des Dokuments =========================
% Stellt die Ränder der Seiten ein
\usepackage[top=4cm, bottom=3cm]{geometry}

% Stellt die Möglichkeit zum Spezifizieren und Definieren von Farben zur Verfügung.
\usepackage{xcolor}
\definecolor{HSTmint}{HTML}{8FD6BD}     % Mint-Farbton der Hochschule-Trier
\definecolor{HSTorange}{HTML}{BE531C}   % Orange-Farbton der Hochschule-Trier
\definecolor{HSTgelb}{HTML}{D9C756}     % Senfgelb-Farbton der Hochschule-Trier
\definecolor{HSTpetrol}{HTML}{115E67}   % Petrol-Farbton der Hochschule-Trier

% Stellt das Aussehen und die Verwendung von Verlinkungen ein
\usepackage{hyperref}
\hypersetup{
    colorlinks=true,
    linkcolor=black,
    urlcolor=HSTorange,
    breaklinks=true,
    urlbordercolor=HSTmint,
    pdftitle={\thema\ -\ \autor\ -\ \abgabedatum},
    pdfauthor=\autor,
}

% stellt die chapterthumbs (Schwarze Balken mit Kapitelnummerierung in weiß) am rechten Rand der rechten/ungeraden Seiten zur Verfügung. Dazu wird das nicht supportete Package "chapterthumbs" verwendet, dass im Ordner "styles", dieser Vorlage liegt.
\usepackage{styles/chapterthumb}
%\usepackage{scrlayer-scrpage}  % schon in "chapterthumb" enthalten (wird dort required)
% verwendet die in "Advanced_I" definierten Layout-Design-Vorgaben für die Darstellung der Dokumentation. Die Datei ist eine überarbeitete Version einer Vorlage (mit gleichem Namen), die bereits in der vorherigen Version dieser Vorlage existierte.
\usepackage{styles/Advanced_I}

% Dieses Package stellt die Kapitelnummerieung mit großen (grauen) Zahlen/Buchstaben ein. Dabei können auch Zitate links neben der Kapitelnummerierung eingeblendet werden.
\usepackage[avantgarde]{quotchap}

%erlaubt das Hinzufügen von doppelten horizontalen Linien
\usepackage{hhline}

%erlaubt das Einfügen mehrerer Spalten nebeneinander. Sowohl im Fließtext, als auch in Tabellen
\usepackage{multicol}

%=== Grafikpackages ============================================
% Ermöglicht das Einbinden von Bildern
\usepackage{graphicx}

% Erlaubt es Abbildungen neben dem Fließtext zu platzieren
\usepackage{wrapfig}

% Optionen für Bildunterschriften
\usepackage[format=hang, font={footnotesize,sf}, labelfont={bf}, margin=1cm, aboveskip=5pt, position=bottom]{caption}

% Beschriftung von figures mit subfigures
\usepackage{subcaption}

% bindet [H] zum positionieren ein
\usepackage{float}

% bindet den Befehl \afterpage ein, der es ermöglicht floats (sicher) am Anfang der nächsten Seite zu verwenden. Siehe dazu die Dokumentation des Packages
\usepackage{afterpage}

% Package zur Visualisierung. Ermöglicht das erstellen von Grafiken in LaTeX
\usepackage{tikz}

% Dieses Package ist ein mächtiges Visualisierungs-Werkzeug, dass gut geeignet ist um wissenschaftliche/technische Grafiken zu erstellen. ACHTUNG: Jeder der so erstellten Grafiken kann die Render-/Erstellungszeit des Dokuments massiv erhöhen.
\usepackage{pgfplots}
\pgfplotsset{width=10cm,compat=1.9}
% Die folgenden beiden Zeilen sorgen dafür, dass die mit tikz oder pgfplots erstellten Grafiken extern (in anderen PDF-Dateien) gespeichert werden. Somit kann die Render-/Erstellungszeit für die Hauptdatei reduziert werden.
\usepgfplotslibrary{external}
\tikzexternalize

% Erlaubt es PDF-Dateien in der eigenen Datei einzubinden.
\usepackage{pdfpages}

%=== Tabellen-Packages =========================================
% erweiterte Optionen für Tabellen
\usepackage{array}

% implementiert eine Umgebung zur Verwendung von mehrseitigen Tabellen.
\usepackage{longtable}

%=== Code-Auszüge-Packages =====================================
% ermöglicht das Einbinden von Code als Auszug
\usepackage{listings}
% Ändert den Namen von 'Listing' zu 'Code-Auszug'
\renewcommand{\lstlistingname}{Code-Auszug}
% stellt Beschriftungen von Code-Auszügen ein
\captionsetup[lstlisting]{font={small,rm}}
% Ändert den Namen des Code-Verzeichnisses von 'Listings' zu 'Code-Auszugs-Verzeichnis'
\renewcommand\lstlistlistingname{Code-Auszugs-Verzeichnis}

%=== Auflistungs-Packages ======================================
%bindet \outline ein um Nummerierungen schön formatiere verwenden zu können
\usepackage{outlines}

%=== Zitier-Packages ===========================================
% erlaubt das Angeben von Zitaten, direkt gekoppelt mit der Quellenangabe (unter Anderem \cite)
\usepackage{csquotes}
% bindet Zitierstyle nach DIN ein.
\usepackage[numbers,round, sort]{natbib}

%=== Mathematik-Pakete =========================================
% ermöglicht es vergleichsweise einfach SI-konforme Zahlen- und Einheitendarstellung zu verwenden
\usepackage[locale=DE]{siunitx}
% Definiert Umgebungen für mathematische Gleichungen - verwendet das Package "amsmath" als Grundlage --> Übergabeparameter können in Dokumentation von amsmath nachgeschlagen werden
\usepackage[intlimits]{empheq}
% ermöglicht es alle mathematischen Symbole der American Mathematical Society zu verwenden
\usepackage{amssymb}
% ermöglicht das Stilisieren von Symbolen um beispielsweise die Symbole der LaPlace oder Fourier-Transformation darzustellen
\usepackage{mathrsfs}
% erlaubt das Stilisieren von Symbolen, um bspw. die Zeichen für natürliche Zahlen, ganze Zahlen etc. zu generieren.
\usepackage{bbm}

% Hier wird für verschiedene Programmiersprachen die Darstellung des Codes, wenn dieser in der Ausarbeitung eingebunden wird, eingestellt. Damit die Darestellung auch für den eigenen Code korrekt ist müssen Änderungen an den Schlüsselwörtern/keywords vorgenommen werden.

% Mit der folgenden Zeile können Standard-Definitionen für die Darstellung verschiedener Programmiersprachen geladen werden. Siehe dazu die Dokumentation zum Package "listings"
%\lstloadlanguages{Python, C++}

%%%%%%%%%%%%%%%%%%%%%%%%%%%%%%%%%%%%%%%%%%%%%%%%%%%%%%%%%%%%%%
% ES FOLGT DIE DEFINITON DER PYTHON-DARSTELLUNG
%%%%%%%%%%%%%%%%%%%%%%%%%%%%%%%%%%%%%%%%%%%%%%%%%%%%%%%%%%%%%%
% Definition der Farben für die Darstellung des Python-Codes
	\definecolor{py_bak_col}{HTML}{19232D}
	\definecolor{py_com_col}{HTML}{62A192}
	\definecolor{py_str_col}{HTML}{B0E686}
	\definecolor{py_key_col}{HTML}{C670E0}
	\definecolor{py_self_col}{HTML}{EE6772}
	\definecolor{py_buin_col}{HTML}{FAB16C}
	\definecolor{py_def_col}{HTML}{57D6E4}

% Definition der Darstellung des Python-Codes
  	\lstdefinelanguage{pyhaff}[]{Python}
  	{
  	sensitive=false,
  	morecomment=[l]{\#},
	morestring=[s][\color{py_str_col}]{"}{"},
	morestring=[s][\color{py_str_col}]{f"}{"},
	basicstyle=\rmfamily\selectfont\color{white},
	classoffset=0,
	morekeywords={for, in, if, elif, else, while, pass, not, and, or, return, def, class, from, import},keywordstyle={\color{py_key_col}\bfseries},
  	classoffset=1,
  	morekeywords={len,range},keywordstyle=\color{py_buin_col},
  	classoffset=2,
  	morekeywords={self},keywordstyle=\color{py_self_col},
  	classoffset=3,
  	morekeywords={add_select_field, send_html_2_user, create_exercise},keywordstyle=\color{py_def_col},
  	classoffset=0,
	%identifierstyle={\color{Navy}},
  	commentstyle={\color{py_com_col}},
  	%morecomment=[s][\color{darkgray}]{/*}{*/}
  	stringstyle=\color{py_str_col},
  	%morestring=[s][\color{py_str_col}]{'}{'},
  	backgroundcolor={\color{py_bak_col}},
  	breaklines=true,
  	columns=flexible,
  	frame=single,
  	framesep=6pt,
  	xleftmargin=11mm,
  	framexleftmargin=8mm,
  	float=hb,   %Einstellung für Platzierung des Objekts: h=genau an dieser Stelle, b=bottom, unteres Ende der Seite
  	numbers=left,
  	rulecolor=\color{white},
  	captionpos=b, % Position des Kommentars, hier unterer Rand.
  	escapeinside={(*@}{@*)},
  	}

%%%%%%%%%%%%%%%%%%%%%%%%%%%%%%%%%%%%%%%%%%%%%%%%%%%%%%%%%%%%%%
% ES FOLGT DIE DEFINITON DER C-DARSTELLUNG
%%%%%%%%%%%%%%%%%%%%%%%%%%%%%%%%%%%%%%%%%%%%%%%%%%%%%%%%%%%%%%
% Definition der Farben für die Darstellung des C-Codes
	\definecolor{cpp_com_col}{HTML}{7F007F}
	\definecolor{cpp_str_col}{HTML}{017F01}
	\definecolor{cpp_idn_col}{HTML}{40409F}
	\definecolor{cpp_key_col}{HTML}{587AE4}

% Definition der Darstellung des C-Codes  	
	\lstdefinelanguage{Chaff}[]{C++}
	{
	basicstyle=\rmfamily,%\small,
	keywordstyle={\color{cpp_key_col}\bfseries},
	identifierstyle={\color{cpp_idn_col}},
	commentstyle={\color{cpp_com_col}},
	%morecomment=[s][\color{darkgray}]{/*}{*/}
	stringstyle=\color{cpp_str_col},
	morestring=[s][\color{cpp_str_col}]{"}{"},
	backgroundcolor={\color{white}},
	breaklines=true,
	columns=flexible,
	frame=single,
	framesep=6pt,
	xleftmargin=11mm,
	framexleftmargin=8mm,
	float=hb,   %Einstellung für Platzierung des Objekts: h=genau an dieser Stelle, b=bottom, unteres Ende der Seite
	numbers=left,	
	captionpos=b, % Position des Kommentars, hier unterer Rand.
	escapeinside={(*@}{@*)}
	}
%%% Hier werden neue LaTeX-Befehle definiert, die im Dokument verwendet werden

% aktiviert die chapterthumbs (Schwarze Balken mit Kapitelnummerierung in weiß) am rechten Rand der rechten/ungeraden Seiten. Dazu wird das nicht supportete Package "chapterthumbs" verwendet, dass im Ordner "styles", dieser Vorlage liegt.
\newcommand{\kapitelbalken}[1]{\ifnum\value{#1}=1 \lohead[\putchapterthumb]{\putchapterthumb} \else\lohead[]{} \fi}

% definiert den Befehl \TEX, um das Symbol von TEX-korrekt darstellen zu können
\newcommand{\TEX}{$\mathrm{T\!_{\displaystyle E} \! X}$}

% definiert den Befehl \LATEX2e, um das Symbol für LATEX2epsilon korrekt darzustellen.
\newcommand{\LaTeXze}{\LaTeX $2_\varepsilon$}

% definiert einen neuen Spaltentyp für Tabellen
\newcolumntype{C}{>{\centering\arraybackslash}m{2cm}}

% definiert ein hochgestelltes 'textregistered'-Symbol
\newcommand{\registered}{$^{\tiny{\text{\textregistered}}}$}


% Anfang des Dokuments --> Hier wird der Inhalt der Seiten definiert
% Zunächst wird der Teil vor dem eigentlich Inhalt eingebunden.
% =======================================================================================
\begin{document}
% Stellt die Seitennummerierung auf römische Zählung um
\pagenumbering{Roman}

% Titelseite
% In dieser Datei werden Aussehen und Inhalt der Titelseite der Dokumentation festgelegt
% Dabei sollten höchstens die Zeilen mit "\betrieb{}" und "\betreuer" verändert werden. Diese sollen nur dann eingefügt werden, wenn es sich um eine externe (nicht-hochschul-interne) Arbeit handelt.

\thispagestyle{plain}
\begin{titlepage}
\begin{addmargin}[-1cm]{-2cm} % Gewährleistet eine  manuelle Anpassung der Seitenränder für die Titelseite, damit wegen der ungleichen Seitenränder, die durch das zweiseitige Dokument entstehen, die Titelseite richtig zentriert werden kann
	\begin{center}
		\includegraphics[width=\textwidth]{images/eTRainer-Logo.png}
		\vspace{1cm}\rule{\textwidth}{0.04cm}	
		\vspace{0.5cm}\LARGE{\art{}}\\
		%\vspace{0.1cm}\large{im Unternehmen \betrieb{}}\\
		\vspace{1cm}\LARGE\textbf{{\thema{}}}\\
		\vspace{1cm}\LARGE\textbf{\themaEnglisch{}}\\
		\vspace{2cm}\large{\autor{}}\\
		\vspace{0.25cm}\large{Mat.Nr.: \matrikel{}}\\
		\vspace{1cm}Betreuer:\\
		\vspace{0.2cm}\prof{}\\
		%\vspace{0.1cm}\betreuer{}\\
		\vspace{1cm}Datum:\\
		\vspace{0.2cm}\abgabedatum%\today
	\end{center}
\end{addmargin}
\end{titlepage}

\newpage 
\thispagestyle{empty}
\quad 
\newpage
\thispagestyle{scrheadings}
% Beginnt Seitenzahl bei 1
\setcounter{page}{1}

% Eidesstattliche Erklärung
% In dieser Datei werden Aussehen und Inhalt der Eidesstattlichen Erklärung definiert. Dabei sollte hier höchstens der Text der Erklärung verändert werden.

\begin{addmargin}[0cm]{0cm}
		\begin{center}
			\LARGE{\textbf{Eidesstattliche Erklärung}}\\
			\rule{\textwidth}{0.04cm}	
		\end{center}
		Ich versichere hiermit, dass ich die vorliegende Arbeit selbstständig verfasst habe und keine anderen als die im Literaturverzeichnis angegebenen Quellen benutzt habe.
		Alle Stellen, die wörtlich oder sinngemäß aus veröffentlichten oder noch nicht veröffentlichten Quellen entnommen sind, sind als solche kenntlich gemacht.
		Die Zeichnungen oder Abbildungen in dieser Arbeit sind von mir selbst erstellt worden oder mit einem entsprechenden Quellennachweis versehen.
		Diese Arbeit ist in gleicher oder ähnlicher Form noch bei keiner anderen Prüfungsbehörde eingereicht	worden.
		\begin{center}
			\vspace{1.5cm}
			\begin{tabular}{lp{2em}l}
				\hspace{5cm}   && \hspace{5cm} \\
				\cline{1-1}\cline{3-3}
				Ort, Datum  && \hfill Unterschrift
			\end{tabular} 
		\end{center}
\end{addmargin}
\cleardoublepage

% Danksagungen
% In dieser Datei werden Aussehen und Inhalt der Danksagung definiert. Dabei sollte hier höchstens der Text der Danksagung verändert werden.

\begin{addmargin}[0cm]{0cm}
		\begin{center}
			\LARGE{\textbf{Danksagung}}\\
			\rule{\textwidth}{0.04cm}	
		\end{center}		
Ich bedanke mich bei all denjenigen, die mich während der Anfertigung dieser Arbeit unterstützt und motiviert haben. 
Ein ganz besonderer Dank gilt meinem Betreuer Prof. Dr. rer. nat. E.-G. Haffner. 
Ebenfalls möchte ich der Projektgruppe eTRainer der Hochschule Trier, sowie meinen Korrekturlesenden danken.
\end{addmargin}
\cleardoublepage

% Abstract
% In dieser Datei werden Aussehen und Inhalt des "Abstract" definiert.

%%% Abstract auf Englisch
\begin{addmargin}[0cm]{0cm}
		\begin{center}
			\LARGE{\textbf{Abstract}}\\
			\rule{\textwidth}{0.04cm}	
		\end{center}

This is a summary of all the important points and achivements of this work.

% Trennung zwischen dem englischen und dem deutschen Abstract.
\vspace{1cm}  

%%% Abstract auf Deutsch
\begin{center}
			\LARGE{\textbf{Zusammenfassung}}\\
			\rule{\textwidth}{0.04cm}	
		\end{center}

Hierbei handelt es sich um eine Zusammenfassung aller wichtigne Punkte und Errungenschaften dieser Arbeit.

\end{addmargin}
\cleardoublepage

% Abkürzungsverzeichnis
% In dieser Datei wird der Inhalt des Abkürzungsverzeichnisses definiert.

\chapter*{Abkürzungsverzeichnis}
\begin{longtable}{ll}
\textbf{Adam} & Adaptive Moment Estimation \\
\textbf{CT} & Computertomographie \\
\textbf{CNN} & Convolutional Neural Network \\
\textbf{DL} & Deep Learning \\
\textbf{GAN}  & Generative Adversarial Network \\ 
\textbf{IS} & Interception Score \\
\textbf{ML} & Machine Learning\\
\textbf{MRT} & Magnetresonanztomographie \\
\textbf{MS} & Mode Score\\
\textbf{ReLU} & Rectified Linear Units\\
\textbf{ResNET} & Residual Neural Network\\
\textbf{SSIM} & Structural Similarity Index Measure\\
\end{longtable}

\cleardoublepage

% Inhaltsverzeichnis
\tableofcontents
\cleardoublepage

% beginnt mit Seitennummerierung, arabische Zahlen und speichert die bisherige Zählung in römischen Ziffern in der Variable Romanpagecount ab --> wird in dieser Version der Vorlage nicht weiter verwendet. Er könnte aber verwendet werden, um bspw. die Seiten der Verzeichnisse am Ende der Dokumentation in römischen Zahlen mit gleicher Nummerierung weiter zählen zu lassen.
\newcounter{Romanpagecount}
\setcounter{Romanpagecount}{\value{page}}
\pagenumbering{arabic}

% aktiviert die chapterthumbs (Schwarze Balken mit Kapitelnummerierung in weiß) am rechten Rand der rechten/ungeraden Seiten. Dazu wird das nicht supportete Package "chapterthumbs" verwendet, dass im Ordner "styles", dieser Vorlage liegt.
\kapitelbalken{kb}


% Anfang des Dokuments --> Hier wird der Inhalt der Seiten definiert
% =======================================================================================

\chapter{Einleitung}
Hier wird in die Arbeit eingeleitet.
\chapter{Grundlagen}

\section{Generative Adversarial Networks}
Generative Adversarial Networks, kurz GANs, sind eine aufstrebende Technologie im Bereich des maschinellen Lernens und der künstlichen Intelligenz. Inspiriert von Ian Goodfellow und seinen Kollegen im Jahr 2014, bieten GANs eine effiziente Möglichkeit, tiefe Repräsentationen von Daten zu erlernen, ohne dass große Mengen an annotierten Trainingsdaten benötigt werden\cite{Creswell.2018}. 
Dies wird durch die Verwendung von Backpropagation und den Wettbewerb zwischen zwei neuronalen Netzen - dem Generator und dem Diskriminator - erreicht. 
Daraus ergeben sich zahlreiche neue Ansätze zur Generierung realistischer Inhalte. 
Die Anwendungen reichen von der Bildgenerierung bis hin zur Superauflösung und Textgenerierung \cite{Aggarwal.2021}.

\subsection{Funktionsweise}
Der Generator und der Diskriminator sind die Hauptkomponenten eines GAN. Die beiden neuronalen Netze werden gleichzeitig trainiert und konkurrieren miteinander, wobei der Generator versucht, den Diskriminator zu täuschen, indem er synthetische Inhalte erzeugt. Um die Glaubwürdigkeit des Generators zu erhöhen, so dass der Diskriminator nicht mehr zwischen den Eingaben unterscheiden kann, wird das gesamte Netz trainiert. Die Netze werden in der Regel als mehrschichtige Netzwerke implementiert, die aus Convolutional und Fully-Connected Schichten bestehen\cite{Creswell.2018}.

\subsubsection*{Generator}
Der Generator dient zur Erzeugung künstlicher Daten wie Bilder und Texte. 
Der Generator ist nicht mit dem realen Datensatz verbunden und lernt daher nur durch die Interaktion mit dem Diskriminator. Wenn der Diskriminator nur noch 50\% der Eingaben richtig vorhersagt, gilt der Generator als optimal\cite{Creswell.2018}.

\subsubsection*{Diskriminator}
Die Unterscheidung zwischen echten und unechten Eingaben ist Aufgabe des Diskriminators. Der Diskriminator kann sowohl künstliche als auch reale Daten verwenden. 
Wenn der Diskriminator nicht mehr richtig unterscheiden kann, wird er als konvergierend bezeichnet\cite{Aggarwal.2021}. Andernfalls wird er als optimal bezeichnet, wenn seine Klassifizierungsgenauigkeit maximiert ist. Im Falle eines optimalen Diskriminators wird das Training des Diskriminators gestoppt und der Generator trainiert alleine weiter, um die Genauigkeit des Diskriminators wieder zu verbessern\cite{Creswell.2018}.

\subsection{Training}
Das Training besteht in der Optimierung der Parameter für sowohl den Generator als auch den Diskriminator durch die Anwendung von Backpropagation zur Verbesserung dieser Parameter. Dieses Verfahren wird häufig als anspruchsvoll und instabil beschrieben. Einerseits gestaltet sich die Herausforderung, beide Modelle konvergieren zu lassen. Andererseits besteht die Problematik darin, dass der Generator Muster erzeugen kann, die für verschiedene Eingaben äußerst ähnlich sind, was als "Mode-Collapse-Problem" bekannt ist. Der Diskriminatorverlust kann sich rasch gegen Null konvergieren, wodurch ein zuverlässiger Gradientenfluss für die Aktualisierung des Generators verhindert wird.
Zur Bewältigung dieser Herausforderungen wurden verschiedene Lösungsansätze vorgeschlagen. Ein Beispiel ist die Verwendung heuristischer Verlustfunktionen. Eine alternative Strategie, die von Sonderby et al. vorgeschlagen wurde, besteht darin, den Datensatz vor der Verwendung zu verrauschen \cite{Creswell.2018}.

\subsubsection*{Adversarieller Verlust}
Der adversarielle Verlust, auch als GAN-Verlust bekannt, spielt eine zentrale Rolle im Trainingsprozess. Dieser Verlust basiert auf dem Konzept des Minimax-Spiels zwischen dem Generator und dem Diskriminator. Der Generator strebt danach, den Diskriminator zu überlisten und Daten zu erzeugen, die von echten Daten nicht zu unterscheiden sind. Gleichzeitig ist es das Ziel des Diskriminators, zwischen echten und generierten Daten zu differenzieren. Der adversarielle Verlust wird in der Gleichung 2.1 repräsentiert.

\begin{equation}
    \min_G \max_D V(D, G) = \mathbb{E}_{x\sim p_{\text{data}}(x)}[\log D(x)] + \mathbb{E}_{z\sim p_z(z)}[\log(1 - D(G(z)))]
\end{equation}

Hierbei bezeichnet $G$ den Generator, $D$ den Diskriminator, $x$ echte Daten, $z$ das Rauschen und $p_{data}$ sowie $p_z$ die Wahrscheinlichkeitsverteilungen von echten Daten und Rauschen. Der Minimax-Ansatz impliziert, dass der Generator versucht, den Verlust zu minimieren, während der Diskriminator versucht, ihn zu maximieren. Eine gezielte Optimierung und Anpassung des adversariellen Verlusts ist entscheidend, um Herausforderungen wie dem Mode-Collapse-Problem und den Konvergenzproblemen zu begegnen \cite{Hong.2020}.

\subsection{Anwendungen}
GANs wurden ursprünglich im Kontext des unüberwachten maschinellen Lernens konzipiert, haben jedoch herausragende Leistungen sowohl im halbüberwachten Lernen als auch im Reinforcement Learning gezeigt \cite{Aggarwal.2021}. Diese Vielseitigkeit hat dazu geführt, dass GANs in verschiedenen Domänen wie dem Gesundheitswesen und dem Bankwesen Anwendung finden.

Im Bereich der medizinischen Bildgebung bieten GANs innovative Lösungsansätze, um den Herausforderungen von Datenknappheit und Patientenprivatsphäre zu begegnen. Von der Erkennung und Behandlung chronischer Krankheiten über die Segmentierung bis hin zur Bildrekonstruktion können GANs vielfältige Anwendungen haben. Zahlreiche GAN-Ansätze wurden bereits entwickelt, um die Rauschunterdrückung in medizinischen Bildverfahren zu verbessern, was wiederum die Qualität von Diagnosen steigern kann \cite{Yi.2019}. Darüber hinaus werden GANs nicht nur im Gesundheitswesen, sondern auch in anderen Bereichen eingesetzt, insbesondere in der Bildsynthese. 

In der Finanzindustrie helfen GANs, das Handels- und Risikomanagement zu verbessern, indem sie synthetische Zeitreihen erzeugen, die wichtige Finanzdaten widerspiegeln \cite{Eckerli.2021}. Zheng et al. (2018) schlugen beispielsweise eine GAN vor, die auf dem Telekommunikations-Betrugsfall in China im Jahr 2017 basiert und die Wahrscheinlichkeit berechnet, wann eine Überweisung betrügerisch sein könnte \cite{Zheng.2018}. 

Des Weiteren wurden Forschungsanstrengungen unternommen, um mittels GANs menschliche Bewegungen vorherzusagen, insbesondere anhand von 3D-Skelett-\\sequenzen \cite{Jain.2020}. Zudem ermöglichen GANs die Identifikation von 3D-Objekten sowie die Generierung realistischer Bilder und Texte in verschiedenen Anwendungsbereichen \cite{Aggarwal.2021}.
Die Vielseitigkeit von GANs eröffnet somit ein breites Spektrum an Potenzialen für verschiedene Anwendungsbereiche.

\subsection{Limitationen}
Ein kritisches Problem von GANs ist die Instabilität des Trainings aufgrund von Mode-Collapse, was die Weiterentwicklung des generativen Lernens und potentielle Anwendungen einschränkt\cite{Liu.2022}. Der Generator lernt nur Bilder bestimmter Arten der Datenverteilung.  Andere Arten, die ebenfalls in der Verteilung vorkommen, werden hingegen vernachlässigt\cite{Srivastava.2017}. Ansätze wie das Hinzufügen von Rauschen zum Netzwerk, eine Manifold Entropy Estimation \cite{Liu.2022} und implizites Variationslernen \cite{Srivastava.2017} wurden bereits vorgeschlagen, um dieses Problem zu lösen.
Des Weiteren birgt die Fähigkeit eines GANs zur Generierung von Inhalten, die nahezu identisch mit authentischen Inhalten sind, potenzielle Herausforderungen in realen Szenarien, insbesondere im Zusammenhang mit der menschlichen Bildsynthese. Diese Fähigkeit ermöglicht es Betrügern, gefälschte Profile in sozialen Medien zu erstellen. Gezielte Anwendungen von GANs, die darauf ausgelegt sind, einzigartige und realistische Bilder von Personen zu erzeugen, die in der Realität nicht existieren, könnten die Erstellung falscher Profile erschweren\cite{Aggarwal.2021}.


\section{Pix2Pix}
Pix2Pix hat sich als zentrales Framework für Bild-zu-Bild-Übersetzungen auf der Basis von bedingten generativen adversariellen Netzwerken (cGANs) etabliert. Es ermöglicht die Erstellung einer abstrakten Abbildung von einem Eingangsbild zu einem korrespondierenden Ausgangsbild und bewältigt dabei eine vielfältige Palette an Bildübersetzungsaufgaben, wie die Transformation von Skizzen in realistische Bilder oder die Konvertierung von Tages- zu Nachtaufnahmen.
  

\subsection{Pix2Pix-Kernkonzepte}
\subsubsection{Generator}
Die Bildverarbeitung hat in den letzten Jahren durch den Einsatz tiefer neuronaler Netzwerke erhebliche Fortschritte gemacht. Im Mittelpunkt vieler dieser Fortschritte steht die U-Net-Architektur, die speziell für die Bildsegmentierung entwickelt wurde. Diese Architektur zeichnet sich durch ihre angeklügelte Kombination aus Encoder- und Decoder- Strukturen sowie durch den Einsatz von Skip-Verbindungen aus \cite{PhillipIsola.}. 
 \newline
Bei der Encoder-Decoder-Struktur handelt es sich um einen Ansatz, bei dem das Eingangsbild zunächst durch den Encoder schrittweise reduziert wird. Dieser Prozess dient dazu, wesentliche Merkmale des Bildes zu erfassen. Anschließend wird das Bild durch den Decoder wiederhergestellt, indem die zuvor extrahierten Merkmale verwendet werden. Während dieser Prozesse besteht jedoch das Risiko des Informationsverlustes, insbesondere in den tieferen Schichten des Netzwerks.
Um dieses Problem zu adressieren, führt die U-Net-Architektur Skip-Verbindungen ein. Diese direkten Verbindungen zwischen korrespondierenden Schichten des Encoders und Decoders sorgen dafür, dass Detailinformationen nicht verloren gehen. Genauer gesagt, ermöglichen diese Verbindungen den direkten Informationsfluss zwischen jeweils äquivalenten Schichten, wodurch die Rekonstruktion des Bildes im Decoder mit einer höheren Genauigkeit erfolgt\cite{PhillipIsola.}. \newline
Die Bedeutung von Skip-Verbindungen zeigt sich insbesondere in Anwendungen wie der Bild-zu-Bild-Übersetzung. Hier muss oft ein Bild mit niedriger Auflösung in ein Bild mit hoher Auflösung überführt werden, ohne dass Details verloren gehen. Die U-Net-Architektur, die angereichert mit diesen Verbindungen ist, ermöglicht daher eine feinere Rekonstruktion, die sowohl globale als auch lokale Informationen berücksichtigt \cite{PhillipIsola.}.  \newline
Somit kann die U-Net-Architektur durch ihre Kombination aus Encoder-Decoder-Struktur und Skip-Verbindungen ein effektives Werkzeug für die Bildsegemtierung darstellen. Ihre Fähigkeit, sowohl globale Muster als auch feine Details zu berücksichtigen, macht sie zu einer bevorzugten Wahl für viele Bildverarbeitungsaufgaben \cite{PhillipIsola.}. \newline
In Abbildung \ref{fig:unet} ist die typische U-Net-Architektur dargestellt. Die linke Seite des ''U'' repräsentiert den Encoder-Teil, der das Eingangsbild schrittweise reduziert und wesentliche Merkmale extrahiert. Die rechte Seite repräsentiert den Decoder-Teil, der das Bild mithilfe der extrahierten Merkmale rekonstruiert. Die horizontalen Linien repräsentieren die Skip-Verbindungen, die sicherstellen, dass Detailinformationen zwischen den korrespondierenden Schichten des Encoder und Decoders direkt übertragen werden \cite{PhillipIsola.}. \newline 
In der Pix2Pix Technologie dient diese U-Net-Architektur als Generator. Er ist das zentrale Element, das für die Bild-zu-Bild-Übersetzung verantwortlich ist. Die Wahl der U-Net-Struktur für den Generator liegt in ihrer Fähigkeit, feinere Details und Kontextinformationen aus dem Eingangsbild beizubehalten, was für die Bild-zu-Bild-Übersetzung von entscheidender Bedeutung ist. Die Encoder-Decoder-Struktur des U-Net ermöglicht es dem Generator, den globalen Kontext des Bildes zu erfassen, während die Skip-Verbindungen sicherstelle, dass auch lokale Details im resultierenden Bild berücksichtigt werden \cite{PhillipIsola.}.

\begin{figure}[h]
	\centering
	\includegraphics[width=0.7\linewidth]{./images/unet.png}
	\caption{Schematische Darstellung der U-Net-Architektur. Die Architektur besteht aus einem Encoder-Teil (links), einem Decoder-Teil (rechts) und Skip-Verbindungen zwischen korrespondierenden Schichten.}
	\label{fig:unet}
\end{figure}

\subsubsection{Diskriminator}

Im adversariellen Lernprozess spielen Generatoren und Diskriminatoren eine entscheidende Rolle. Während der Generator versucht, Daten zu erzeugen, die von echten Daten kaum zu unterscheiden sind, evaluiert der Diskriminator die vom Generator erzeugten Daten und versucht, zwischen echten und gefälschten Daten zu unterscheiden \cite{PhillipIsola.}.\newline
Im Kontext von Generative Adverserial Networks (GANs), insbesondere im speziellen Fall des Pix2Pix GANs, spielt der PatchGAN-Diskriminator eine besonders wichtige Rolle. Der zentrale Unterschied dieses Diskriminators zu traditionellen Diskriminatoren liegt in der Art und Weise, wie er Bilder bewertet. Statt das gesamte Bild zu beurteilen, zerlegt der PatchGAN-Diskriminator das Bild in mehrere kleinere Bildabschnitte oder Patches und bewertet jeden Patch einzeln auf seine Echtheit \cite{PhillipIsola.}. \newline
Ein solches Vorgehen hat den klaren Vorteil, dass feinere Strukturen und Details in den Bildern erkannt und beurteilt werden können. Durch diese segmentierte Beurteilung kann der Diskriminator besser einschätzen, ob die Struktur und Beschaffenheit eines bestimmten Bildteils realistisch ist. Dies ist besonders nützlich, da kleinere Unstimmigkeiten in den Bildern, die ein traditioneller Diskriminator möglicherweise übersieht, vom PatchGAN erfasst werden können. \newline
Ein weiterer Vorteil des PatchGAN-Diskriminators ist seine Skalierbarkeit. Da er auf festen Patchgrößen basiert, kann er flexibel auf Bilder unterschiedlicher Größen angewendet werden, ohne dass das zugrunde liegende Modell geändert werden muss. Dies führt nicht nur zu einer schnelleren Bildverarbeitung, sondern ermöglicht auch eine effiziente Ausführung auf großen Bildern. Darüber hinaus reduziert es potenzielle Kachelartefakte, die bei traditionellen Diskriminatoren auftreten können \cite{PhillipIsola.}.\newline
Eine Metrik, die oft verwendet wird, um die Leistung des Diskriminator zu beurteilen, ist der FCN-Score. Dieser bewertet die Qualität der vom Generator erzeugten Bilder. Ein hoher FCN-Score zeigt, dass der Diskriminator erfolgreich echte von gefälschten Bildern unterscheiden kann. \newline
Der PatchGAN-Diskriminator kann wenn er effektiv eingesetzt wird, zu besseren und realistischeren Bildern im adverseriellen Lernprozess beitragen. Seine Fähigkeit, lokale Bildinformationen zu bewerten, ermöglicht es auch subtile Unterschiede in den Bildern zu erkennen, was zu einer verbesserten Qualität der generierten Bilder führt \cite{PhillipIsola.}.

\subsubsection{L1-Verlustfunktion}

Die L1-Verlustfunktion, auch bekannt als Mean Absolute Error (MAE), spielt eine entscheidende Rolle im Pix2Pix-Modell, einem bedingten Generative Adversarial Network (cGAN) für Bild-zu-Bild-Übersetzungen. Diese Verlustfunktion misst den durchschnittlichen absoluten Unterschied zwischen den vorhergesagten und den tatsächlichen Werten, wodurch sie die Genauigkeit der generierten Bilder verbessert, insbesondere im Hinblick auf die niedrigen Frequenzen im Bild. Die L1-Verlustfunktion trägt somit maßgeblich zur Bewahrung der strukturellen Integrität und des Kontexts des Bildes bei \cite{PhillipIsola.}. \newline
Die Verwendung des L1-Verlusts zusätzlich zum adversariellen Verlust im Pix2Pix-Modell ist von entscheidender Bedeutung. Während der adversarielle Verlust darauf abzielt, die generierten Bilder realistisch erscheinen zu lassen, konzentriert sich der L1-Verlust auf die Genauigkeit der niedrigen Frequenzen, um die strukturelle Integrität und den Kontext des Bildes zu bewahren. Diese Kombination ermöglicht es, sowohl die niedrigen als auch die hohen Frequenzen im Bild effektiv zu erfassen, was zu generierten Bildern führt, die sowohl strukturell korrekt als auch visuell ansprechend sind \cite{PhillipIsola.}. \newline
Die L1-Verlustfunktion neigt jedoch dazu, bei den hohen Frequenzen unscharfe Ergebnisse zu liefern. Dies liegt daran, dass der L1-Verlust den Median der möglichen Werte bevorzugt, was zu einer Glättung der Bildtexturen führen kann. Um dieses Problem zu adressieren und scharfe, hochfrequente Details im Bild zu erhalten, wird der L1-Verlust im Pix2Pix-Modell mit einem adversariellen Verlust kombiniert. Diese synergetische Kombination von Verlustfunktionen ermöglicht es dem Pix2Pix-Modell, hochwertige Bild-zu-Bild-Übersetzungen durchzuführen, die sowohl visuell ansprechend als auch strukturell korrekt sind \cite{PhillipIsola.}. \newline
Darüber hinaus hat sich die Kombination von L1-Verlust und adversariellen Verlust im Pix2Pix-Modell als nützlich für eine Vielzahl von Bild-zu-Bild-Übersetzungsproblemen erwiesen, einschließlich semantischer Segmentierung und Farbgebung. Durch die effektive Erfassung sowohl der niedrigen als auch der hohen Frequenzen im Bild trägt das Pix2Pix-Modell dazu bei, die Qualität der generierten Bilder zu verbessern und ihre Anwendbarkeit auf verschiedene Probleme zu erweitern \cite{PhillipIsola.}.

\subsubsection{Training}

Der Trainingsprozess von Pix2Pix-Generative Adversarial Networks in der Bild-zu-Bild-Übersetzung geht über die bloße Erlernung der Abbildung von Eingabe- zu Ausgabebildern hinaus. Er umfasst auch das Entwickeln einer maßgeschneiderten Verlustfunktion, die speziell auf diese Art der Bildtransformation abgestimmt ist. Dieser umfassende Ansatz ermöglicht es Pix2Pix, sich flexibel an eine Vielzahl von Problemen anzupassen, die in der Vergangenheit unterschiedliche und spezialisierte Ansätze für die Verlustfunktion erforderten. Dadurch wird die breite Anwendbarkeit und Effektivität des Pix2Pix-Modells in verschiedenen Bildübersetzungsaufgaben deutlich. Pix2Pix benötigt eine spezifische Art von Trainingsdaten, um effektiv zu funktionieren. Die Trainingsdaten bestehen aus Paaren von Bildern, wobei jedes Paar ein Eingabebild und das entsprechende Ausgabebild enthält. Diese Bilder können 1-3 Kanäle aufweisen, was bedeutet, dass das Modell sowohl mit monochromatischen (Graustufen) als auch mit farbigen Bildern (RGB) arbeiten kann. Diese Flexibilität in der Kanalverarbeitung ermöglicht eine breitere Anwendung des Pix2Pix-Modells auf verschiedene Bildtypen. Die Bilder müssen nicht in einer bestimmten Weise vorverarbeitet werden, da das Modell direkt auf den Rohpixeln arbeitet. Diese Flexibilität in der Kanalverarbeitung und die Fähigkeit, direkt auf Rohpixeln zu arbeiten, unterstreichen die Vielseitigkeit des Pix2Pix-Modells. Dies wird weiter durch die sorgfältige Auswahl spezifischer Trainingsparameter und Hyperparameter-Optimierungsstrategien ergänzt. Der Adam-Grandientenoptimierungsalgorithmus ist eine Methode zur Optimierung des maschinellen Lernens, die auf adaptiven Schätzungen niedrigerer Ordnung basiert. Er passt automatisch die Lernrate während des Trainings an und eignet sich besonders gut für Probleme mit großen Datensätzen und/oder vielen Parametern. Durch empirische Tests hat sich herausgestellt das eine initiale Lernrate von 0.0002 und die Momentum-Parameter von $\beta1$ = 0.5 und $\beta$ = 0.999 optimal sind, um die Balance zwischen Lerngeschwindigkeit und Stabilität des Trainingsprozesses zu optimieren. Auch die Wahl einer kleinen Batchgröße, typischerweise 1, spielt eine entscheidende Rolle, um die Trainingseffizienz zu maximieren und qualitativ hochwertige Ergebnisse zu erzielen. Diese spezifischen Einstellungen der Trainingsparameter tragen maßgeblich dazu bei, das Potenzial des Pix2Pix-Modells voll auszuschöpfen.\cite{PhillipIsola.}.\newline
Im Rahmen des Trainingsprozesses von Pix2Pix wird eine iterative Methode verwendet, bei der der Generator und der Diskriminator abwechselnd trainiert werden. Zunächst werden die Trainingsdaten vorbereitet, die aus Paaren von Eingabe- und Zielbildern besteht. Diese Bilder werden auf einen Wertebereich von -1 bis +1 skaliert. Diese Normalisierung ist wichtig, da sie zur Stabilisierung des Trainingsprozesses beiträgt. Sie ermöglicht es dem Modell, mit einem standardisierten Datensatz zu arbeiten, was die Lernrate und die Konvergenzgeschwindigkeit verbessert. Während des Tranings versucht der Generator, Bilder zu erzeugen, die vom Diskriminator nicht von realen Bildern unterschieden werden können. Der Diskriminator hingegen lernt, zwischen echten und vom Generator erzeugten Bildern zu unterscheiden. Eine Schlüsselkomponente dieses Prozesses ist die Verwendung einer zusammengesetzten Verlustfunktion, die sowohl den adverseriellen Verlust (bewertet vom Diskriminator) als auch den L1-Verlust(mittlerer absoluter Fehler zwischen generiertem Bild und Zielbild) umfasst. Dadurch wird der Generator dazu angehalten, realistische Übersetzungen der Eingabebilder zu generieren. Dieses Gleichgewicht zwischen Generator und Diskriminator ist entscheidend für die Effektivität des Pix2Pix-Modells \cite{HazemAbdelmotaalAhmedA.AbdouAhmedF.OmarDaliaMohamedElSebaityKhaledAbdelazeem.2021}.
  



\subsection{Anwendungen von Pix2Pix}
Pix2Pix ist eine fortschrittliche Methode für Bild-zu-Bild-Übersetzungsaufgaben und hat eine breite Palette von Anwendungen in der Bildverarbeitung.\newline
Ein markantes Anwendungsbeispiel ist die Umwandlung von architektonische Entwürfe oder Zeichnungen in realistische Gebäudefotos. Besonders eindrucksvoll ist diese Anwendung beim CMP Facades-Datensatz, wo aus simplen Fassadenzeichnungen detailreiche Gebäudebilder generiert werden. \cite{PhillipIsola.}
\newline
Im Bereich der Fotografie wird Pix2Pix verwendet, um Schwarz-Weiß-Fotos in farbige Bilder zu konvertieren, was besonders bei der Restaurierung alter Fotografien von Bedeutung sein kann.
\newline
Die Transformation von Tagesaufnahmen in Nachtbilder ist eine weitere beeindruckende Leistung von Pix2Pix sowie die Umwandlung von Thermalaufnahmen in Farbfotos.
\newline
Schließlich wird Pix2Pix auch zur Vervollständigung von Fotos mit fehlenden Pixeln verwendet, beispielsweise um unvollständige Bilder, die aus Paris StreetView stammen, zu reparieren und zu vervollständigen.
\newline
Diese vielfältigen Anwendungen zeigen die enorme Flexibilität und die breite Einsatzmöglichkeit von Pix2Pix in der modernen Bildbearbeitung und Computergrafik.
 

\section{CycleGAN}
CycleGAN, das 2017 von Jun-Yan Zhu et al. vorgestellt wurde, stellt eine neue Entwicklung im Bereich des maschinellen Lernens und insbesondere der Bildübersetzung zwischen unpaaren Domänen dar. Es erweitert die Pix2Pix-Architektur durch die Einführung einer Zykluskonsistenz-Verlustfunktion (Cycle Consistency Loss), die sicherstellt, dass das Originalbild nach einem Übersetzungs- und Rückübersetzungszyklus erhalten bleibt. Der Generator $G$ transformiert Bilder aus der Domäne $X$ in die Domäne $Y$, während der Generator $F$ den umgekehrten Prozess durchführt. Diese Transformationen werden ohne gepaarte Trainingsdaten durchgeführt.

\subsection{CycleGAN - Kernkonzepte}
\subsubsection{Architektur der Generatoren}
Die Architektur der Generatoren in CycleGAN spielt eine entscheidende Rolle bei der erfolgreichen Durchführung von Bildübersetzungen zwischen unterschiedlichen Domänen. Typischerweise basieren die Generatoren auf dem Residual-Netzwerk (ResNet)-Ansatz, der für seine Fähigkeit, tiefe neuronale Netzwerke zu trainieren, bekannt ist.
\\
ResNet-Blöcke, bestehend aus Convolutional-Schichten, Normalisierungsschichten und Aktivierungsfunktionen, ermöglichen den Generatoren, komplexe Transformationen durchzuführen. Die Nutzung von ResNet-Blöcken erleichtert auch das Training tiefer Netzwerke, was wichtig ist, um hochwertige Übersetzungen zu erreichen.

\subsubsection{Architektur der Diskriminatoren}
Ebenso wie beim Pix2Pix ist die gängige Architektur für die Diskriminatoren in CycleGAN der PatchGAN. 
%Die Architektur der Diskriminatoren in CycleGAN spielt eine entscheidende Rolle bei der Unterscheidung zwischen echten und generierten Bildern. Die Struktur beeinflusst direkt die Fähigkeit des Modells, realistische Übersetzungen zu erzeugen.
\\
%Eine gängige Architektur für die Diskriminatoren in CycleGAN ist der PatchGAN. Hierbei handelt es sich um eine Netzwerkstruktur, die das Bild in kleine Patches aufteilt und jeden Patch separat klassifiziert. Diese Methode ermöglicht eine feine Unterscheidung zwischen echten und generierten Bildern auf lokaler Ebene.
\\
Die Diskriminatoren bestehen in der Regel aus Convolutional-Schichten, gefolgt von Normalisierungsschichten und Aktivierungsfunktionen. In einigen Implementierungen von CycleGAN wird statt der üblichen Batch-Normalisierung die Instance Normalisierung bevorzugt.
\\
Die Instance Normalisierung ist eine Variante der Normalisierung, die auf Instanzebene durchgeführt wird. Im Gegensatz zur Batch-Normalisierung, bei der die Normalisierung über die gesamte Batch-Dimension erfolgt, betrachtet die Instance Normalisierung jede Instanz oder jedes Bild einzeln. Dies kann besonders vorteilhaft sein, wenn die statistischen Eigenschaften der einzelnen Instanzen variieren.

Die Verwendung von Instance Normalisierung in den Diskriminatoren von CycleGAN kann dazu beitragen, eine stabilere und konsistentere Konvergenz während des Trainings zu erreichen. 

\subsubsection{Training}
Das Training von CycleGAN erfolgt durch einen kompetitiven Prozess. Die Generatoren $G$ und $F$ konkurrieren mit den zugehörigen Diskriminatoren $D_X$ und $D_Y$. $D_X$ versucht, zwischen den von $F$ generierten und den echten Bildern aus $X$ zu unterscheiden, während $D_Y$ die von $G$ generierten Bilder von den echten aus der Domäne $Y$ unterscheidet. Die adversarialen Verluste werden dabei optimiert, um sicherzustellen, dass die generierten Bilder für die Diskriminatoren kaum von den echten zu unterscheiden sind.

\subsubsection{Cycle - Konsistenz}
Der CycleGAN führt zusätzlich einen Cycle-Konsistenz ein. Diese gewährleistet, dass die Übersetzungen zwischen den Domänen sowohl vorwärts ($X$ nach $Y$) als auch rückwärts ($Y$ nach $X$) konsistent sind. Die Cycle-Konsistenz fördert die Generierung qualitativ hochwertiger und konsistenter Übersetzungen.
\\
Die Kernidee besteht darin, dass nach der Übersetzung von $X$ nach $Y$ und zurück nach $X$ das resultierende Bild dem ursprünglichen $X$ entsprechen sollte. Dabei wird die Differenz zwischen dem ursprünglichen Bild $x$ mit dem zyklisch übersetzten Bild $F(G(x))$ mittels dem L1-Verlustfunktion minimiert. 
\\
Diese zyklische Konsistenz stellt sicher, dass die Generatoren konsistente und realitätsnahe Übersetzungen zwischen den Domänen erzeugen.

\subsubsection{Identity - Loss}
Zusätzlich zu adversarialen und zyklischen Verlusten spielt der Identitätsverlust eine bedeutende Rolle im Training von CycleGAN. Der Identitätsverlust zielt darauf ab, sicherzustellen, dass die Transformationen nicht zu stark sind und wichtige Merkmale der Ausgangsbilder bewahrt bleiben. Dieser Verlust wird durch die Verwendung von Bildern der jeweiligen Ausgangsdomäne als Eingabe für den Generator und die Berechnung des Verlusts zwischen den generierten und den Eingangsbildern erreicht.

Die Identitätsverluste für $G$ und $F$ werden als zusätzliche Terme in die Gesamtverlustfunktion integriert. Dieser Mechanismus stellt sicher, dass der Generator nicht unerwünschte Veränderungen vornimmt und hilft, feine Details und Strukturen in den generierten Bildern zu erhalten.

\section{Convolutional Layers}
Convolutional Layers repräsentieren eine fundamentale Komponente innerhalb neuronaler Netzwerke, insbesondere im Kontext der Verarbeitung von Bildinformationen. Diese Schicht nutzt Convolution-Operationen, um durch Faltung von Eingabedaten mit Filterkernen lokale Muster zu identifizieren. Die Filterkerne, üblicherweise in Größen wie 3x3, 7x7 oder 9x9, fungieren als kleine Arrays und dienen der Extraktion spezifischer Merkmale im Bild.

Die Funktionsweise dieser Schicht basiert auf der schrittweisen Verschiebung der Filterkerne über das Eingangsbild. An den jeweiligen Pixelpositionen erfolgt eine präzise elementweise Multiplikation, gefolgt von einer anschließenden Summation aller resultierenden Werte. Diese berechneten Werte werden dann in den korrespondierenden Positionen der Feature Map eingetragen, wie in der Abbildung \ref{fig:convolutionalLayer} veranschaulicht. Durch diesen Prozess gewinnen tiefere Schichten des Netzwerks die Fähigkeit, zunehmend komplexe und abstrakte Informationen auf höheren Ebenen der Hierarchie zu repräsentieren.

\begin{figure}[ht]
	\centering
	\includegraphics[width=0.5\linewidth]{./images/convolutionalLayer.png}
	\caption{Beispiel einer Convolutional-Operation mit einem 3x3 Kernel und Stride 1\cite{Yamashita.2018}}
	\label{fig:convolutionalLayer}
\end{figure}

Die Verwendung verschiedener Kernels, sowohl in Bezug auf Größe als auch Anzahl, erlaubt die Extraktion vielfältiger Merkmale wie Kanten oder Texturen. Diese Flexibilität befähigt das Netzwerk, auf unterschiedlichste visuelle Strukturen ansprechend zu reagieren \cite{Yamashita.2018}.

Zusätzlich fungieren die Kernels als Subsampling-Mechanismus, bedingt durch die begrenzte Ausdehnung der Convolution-Operation bis zum Bildrand. Die Wahl der Strides, als die Distanz zwischen zwei verschobenen Kernelpositionen, beeinflusst diesen Subsampling-Effekt. Die Anwendung von Padding, um das Bild vor jeder Convolution zu vergrößern, kann dazu beitragen, ungewollte Unterabtastung (Downsampling) zu minimieren.

Ein wesentlicher Vorzug von Convolutional Layers besteht in der signifikanten Reduzierung der zu trainierenden Parameter und der Komplexität des Modells. Dies führt zu einer verbesserten Effizienz, da weniger Gewichte optimiert werden müssen, und trägt zur Prävention von Overfitting bei\cite{Yamashita.2018,OShea.2015}.

Die Ergebnisse der Convolutional Layer werden anschließend durch eine nicht-lineare Aktivierungsfunktion weitergereicht, um die Fähigkeit des Netzwerks zur Modellierung komplexer, nicht-linearer Zusammenhänge zu verbessern. Die Einbeziehung einer Aktivierungsfunktion, wie beispielsweise der ReLU (Rectified Linear Unit), ermöglicht es dem Netzwerk, nicht-linear separierbare Muster und Merkmale zu erfassen. Ohne Aktivierungsfunktionen würden die Convolutional Layers nur lineare Transformationen durchführen, was die Lernfähigkeit des Modells deutlich einschränken würde\cite{Sharma.2017}.

\section{Bibliotheken}
\subsection{Tensorflow}
TensorFlow ist ein weit verbreitetes Open-Source-Framework, das für maschinelles Lernen und tiefe neuronale Netzwerke eingesetzt wird\footnote{https://www.tensorflow.org/}. Es bietet eine umfangreiche Plattform zur Entwicklung und Umsetzung von Modellen in verschiedenen Anwendungsbereichen, darunter Bilderkennung und natürliche Sprachverarbeitung. Die Architektur von TensorFlow ermöglicht das Erstellen komplexer maschineller Lernanwendungen und stellt umfassende Tools und Bibliotheken zur Verfügung, um den gesamten Entwicklungsprozess zu unterstützen. Darüber hinaus profitiert TensorFlow von einer aktiven und engagierten Community, die kontinuierlich zur Weiterentwicklung des Frameworks beiträgt\footnote{https://github.com/tensorflow/tensorflow}.

\subsection{Keras}
Ursprünglich als eigenständiges Framework konzipiert und seit TensorFlow 2.0 in die TensorFlow Core API integriert, fungiert Keras als Open-Source-API zur Modellierung von Strukturen im Bereich des Deep Learning\footnote{https://www.tensorflow.org/guide/keras}. Diese Bibliothek, die in der Programmiersprache Python implementiert ist, umfasst sämtliche Phasen des maschinellen Lern-Workflows. Beginnend mit der Datenverarbeitung ermöglicht sie eine Fortführung bis zur präzisen Abstimmung der Hyperparameter während des Trainingsprozesses. Die Keras-Prinzipien zeichnen sich durch Einfachheit, Flexibilität und Leistungsfähigkeit aus und ermöglichen es den Anwendern, die Skalierbarkeit und plattformübergreifenden Fähigkeiten der TensorFlow-Plattform zu nutzen\footnote{https://keras.io/about/}.

\subsection{Matplotlib}
Matplotlib ist eine weit verbreitete und leistungsstarke Python-Bibliothek zur Erstellung von statischen, interaktiven und animierten Visualisierungen. Entwickelt von John D. Hunter im Jahr 2003, hat sich Matplotlib zu einem Standardwerkzeug in der Datenvisualisierung und wissenschaftlichen Forschung entwickelt\footnote{https://matplotlib.org/}.
\chapter{Hauptteil - 1}
Hier wird der Inhalt der Arbeit präsentiert.
\chapter{Hauptteil - 2}
Hier wird der Inhalt der Arbeit präsentiert.
\chapter{Hauptteil - 3}
Hier wird der Inhalt der Arbeit präsentiert.
\chapter{Evaluation}

\section{Vergleich von Pix2Pix und CycleGAN}
- Matching Paare von Bildern sind ebenfalls für das Training nicht nötig (crewall)
- Macht die Datenvorbereitung einfacher und öffnet neue Techniken für Applikationen (crewall)
\chapter{Fazit und Ausblick}
Hier wird ein Fazit und ein Ausblick gegeben.

\section{Fazit}
Das Teamprojekt $"$Analyse, Design und Implementierung von unterschiedlichen Generative Adversarial Network Architekturen im Bereich der Bildverarbeitung$"$ bietet einen umfassenden Einblick in die Welt der Generative Adversarial Networks (GANs) und ihre Anwendungen in der Bildverarbeitung. Bereits in der Einleitung werden verschiedene Aspekte von GANs und ihre Relevanz in der Bildverarbeitung angesprochen. Im weiteren Verlauf der Arbeit wurden diese Aspekte vertieft und konkretisiert, wobei ein solides Fundament aus Fachtermini und Grundlagen im Bereich GANs gelegt wurde. Ein besonderes Augenmerk lag dabei auf den spezialisierten Frameworks Pix2Pix und CycleGAN.

Nicht nur theoretische Konzepte wurden beleuchtet, sondern auch eine praktische Umsetzung erfolgte. Die Modelle Pix2Pix und CycleGAN wurden nicht nur theoretisch diskutiert, sondern auch konkret implementiert und auf einen spezifischen Datensatz trainiert. Die Implementierung und das Training dieser Modelle erfolgten unter Einsatz der Frameworks TensorFlow und Keras, deren Flexibilität und Benutzerfreundlichkeit für die Entwicklung und Optimierung dieser komplexen Modelle entscheidend waren. Durch diesen praktischen Ansatz konnten die theoretischen Konzepte veranschaulicht und die Leistungsfähigkeit sowie die Herausforderungen der verschiedenen GAN-Architekturen in der realen Anwendung demonstriert werden. Die Ergebnisse des Trainingsprozesses bieten wertvolle Einblicke in die Möglichkeiten und Grenzen dieser innovativen Technologien im Bereich der Bildverarbeitung.

%%%======================================================================================

% Anhang
\appendix
\kapitelbalken{akb}
\chapter{Anhang - Code}
Hier sehen Sie den gesamten Quellcode!
\chapter{Anhang - Dokumentationen}
Hier sehen Sie die gesamten Dokumentationen zu den erstellten Programmen.




% muss stehen bleiben, da sonst die Kapitelbalken nicht bei allen Seiten von Anhang B auftauchen
\cleardoublepage

% deaktiviert die chapterthumbs wieder, sodass sie nicht in den Verzeichnissen auftauchen.
\lohead[]{}

% Verzeichnisse
\Ifpdfoutput{\ihead[]{}}{}
% Hier werden alle Verzeichnisse erzeugt, die am Ende der Arbeit aufgelistet werden.
% mit diesem Befehl wird dem Inhaltsverzeichniss das Kapitel "Verzeichnisse hinzugefügt
\addcontentsline{toc}{chapter}{Verzeichnisse}

\addcontentsline{toc}{section}{Literaturverzeichnis}
\nocite{*}						% damit die ganze Literatur im Verzeichniss erscheint, auch, wenn diese nicht zitiert wurde! 
\bibliographystyle{alphadin}	% Stil des Literaturverzeichnisses
\bibliography{literatur}		% Literaturverzeichnis
\cleardoublepage

% Erstellt das Abbildungsverzeichnis
\listoffigures
\addcontentsline{toc}{section}{Abbildungsverzeichnis}   % Abbildungsverzeichnis	
\cleardoublepage

% Erstellt das Tabellenverzeichnis - Bei Bedarf einfügen
\listoftables
\addcontentsline{toc}{section}{Tabellenverzeichnis}		% Tabellenverzeichnis
\cleardoublepage

% Erstellt das Code-Auszugs-Verzeichnis - Bei Bedarf einfügen
\lstlistoflistings
\addcontentsline{toc}{section}{Code-Auszugs-Verzeichnis}% Code-Auszugs-Verzeichnis
\cleardoublepage

% Glossar
\addcontentsline{toc}{section}{Glossar}
\chapter*{Glossar}

\begin{itemize}
	\item \textbf{Keras}:\\ Open-Source-API für Deep Learning, seit TensorFlow 2.0 integraler Bestandteil der TensorFlow Core API.
	\item \textbf{Matplotlib}: \\  Python-Bibliothek zur Erstellung von statischen, interaktiven und animierten Visualisierungen.
	\item \textbf{Python}:\\ Skript- und Programmiersprache, die unter Anderem objektorientiertes Programmieren ermöglicht. 
	\item \textbf{TensorFlow}:\\  Open-Source-Framework für maschinelles Lernen und tiefe neuronale Netzwerke, bekannt für seine Skalierbarkeit und umfangreiche Plattformunterstützung.
\end{itemize} 

% Glossar
\addcontentsline{toc}{section}{Arbeitsverteilung}
\chapter*{Arbeitsverteilung}

\textbf{Teilnehmer 1:  Elisa Du} \\
\\
 Inhalte: \\
\\ Abstract/Zusammenfassung
\\ Grundlagen - Kapitel (exklusiv Pix2Pix)
\\ Literaturreview
\\ Problembeschreibung - Kapitel
\\ Lösungsbeschreibung - Kapitel (exklusiv Implementierung der Pix2PixGAN-Architektur)
\\ Evaluation - Ergebnisse und objektive Bewertung  (exklusiv Pix2Pix)
\\ Evaluation - Vergleich von Pix2Pix und CycleGAN
\\ Anhänge
\\ 
\\
\\
\textbf{Teilnehmer 2: Marcel Hoffmann} \\
\\
 Inhalte: \\
\\ Einleitung
\\ Pix2Pix in Grundlagen - Kapitel
\\ Implementierung der Pix2PixGAN-Architektur in Lösungsbeschreibung - Kapitel
\\ Evaluation - Bewertungskriterien
\\ Evaluation - Pix2Pix Ergebnisse und objektive Bewertung
\\ Evaluation - Vergleich von Pix2Pix und CycleGAN (Theoretischer Vergleich)
\\ Fazit und Ausblick
\\
\\
\\


\end{document}