% Hier werden alle Verzeichnisse erzeugt, die am Ende der Arbeit aufgelistet werden.
% mit diesem Befehl wird dem Inhaltsverzeichniss das Kapitel "Verzeichnisse hinzugefügt
\addcontentsline{toc}{chapter}{Verzeichnisse}

\addcontentsline{toc}{section}{Literaturverzeichnis}
\nocite{*}						% damit die ganze Literatur im Verzeichniss erscheint, auch, wenn diese nicht zitiert wurde! 
\bibliographystyle{alphadin}	% Stil des Literaturverzeichnisses
\bibliography{literatur}		% Literaturverzeichnis
\cleardoublepage

% Erstellt das Abbildungsverzeichnis
\listoffigures
\addcontentsline{toc}{section}{Abbildungsverzeichnis}   % Abbildungsverzeichnis	
\cleardoublepage

% Erstellt das Tabellenverzeichnis - Bei Bedarf einfügen
\listoftables
\addcontentsline{toc}{section}{Tabellenverzeichnis}		% Tabellenverzeichnis
\cleardoublepage

% Erstellt das Code-Auszugs-Verzeichnis - Bei Bedarf einfügen
\lstlistoflistings
\addcontentsline{toc}{section}{Code-Auszugs-Verzeichnis}% Code-Auszugs-Verzeichnis
\cleardoublepage

% Glossar
\addcontentsline{toc}{section}{Glossar}
\chapter*{Glossar}

\begin{itemize}
    \item \textbf{C++}:\\ Hardwarenahe, objektorientierte Programmiersprache.
	\item \textbf{HTML}:\\ Hypertext Markup Language - textbasierte Auszeichnungssprache zur Strukturierung elektronischer Dokumente.
	\item \textbf{HTTP}:\\ Hypertext Transfer Protocol - Protokoll zur Übertragung von Daten auf der Anwendungssicht über ein Rechnernetz.
	\item \textbf{iARS}:\\ innovative Audio Response System - System mit zwei Applikationen (iARS-master-App; iARS-student-App), dass sich zum Einsetzten von e-TR-ainer-Inhalten in Vorlesungen eignet.
	\item \textbf{ISO}:\\ Internationale Vereinigung von Normungsorganisationen.
	\item \textbf{JavaScript}:\\ Skriptsprache zu Auswertung von Benutzerinteraktionen.
	\item \textbf{Konstruktor}:\\ Beim Erzeugen einer Objektinstanz aufgerufene Methode zum Initialisieren von Eigenschaften.
	\item \textbf{MySQL}:\\ Relationales Datenbankverwaltungssystem.
	\item \textbf{OLAT}:\\ Online Learning and Training - Lernplattform für verschiedene Formen von webbasiertem Lernen.
	\item \textbf{OOP}:\\  Objektorientierte Programmierung - Programmierparadigma, nach dem sich die Architektur eine Software an realen Objekten orientiert.
	\item \textbf{Open Source}:\\ Software, die öffentlich von Dritten eingesehen, geändert und genutzt werden kann.
	\item \textbf{PHP}:\\ Skriptsprache zur Erstellung von Webanwendungen.
	\item \textbf{Python}:\\ Skript- und Programmiersprache, die unter Anderem objektorientiertes Programmieren ermöglicht.
	\item \textbf{Shell}:\\ Shell oder auch Unix-Shell - traditionelle Benutzerschnittstelle von Unix-Betriebssystemen
	\item \textbf{Spyder}:\\ Entwicklungsumgebung für wissenschaftliche Programmierung in der Programmiersprache Python.
	\item \textbf{SymPy}:\\ Python-Bibliothek für symbolische Mathematik.
\end{itemize} 