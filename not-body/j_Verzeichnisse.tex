% Hier werden alle Verzeichnisse erzeugt, die am Ende der Arbeit aufgelistet werden.
% mit diesem Befehl wird dem Inhaltsverzeichniss das Kapitel "Verzeichnisse hinzugefügt
\addcontentsline{toc}{chapter}{Verzeichnisse}

\addcontentsline{toc}{section}{Literaturverzeichnis}
\nocite{*}						% damit die ganze Literatur im Verzeichniss erscheint, auch, wenn diese nicht zitiert wurde! 
\bibliographystyle{alphadin}	% Stil des Literaturverzeichnisses
\bibliography{literatur}		% Literaturverzeichnis
\cleardoublepage

% Erstellt das Abbildungsverzeichnis
\listoffigures
\addcontentsline{toc}{section}{Abbildungsverzeichnis}   % Abbildungsverzeichnis	
\cleardoublepage

% Erstellt das Tabellenverzeichnis - Bei Bedarf einfügen
\listoftables
\addcontentsline{toc}{section}{Tabellenverzeichnis}		% Tabellenverzeichnis
\cleardoublepage

% Erstellt das Code-Auszugs-Verzeichnis - Bei Bedarf einfügen
\lstlistoflistings
\addcontentsline{toc}{section}{Code-Auszugs-Verzeichnis}% Code-Auszugs-Verzeichnis
\cleardoublepage

% Glossar
\addcontentsline{toc}{section}{Glossar}
\chapter*{Glossar}

\begin{itemize}
	\item \textbf{Python}:\\ Skript- und Programmiersprache, die unter Anderem objektorientiertes Programmieren ermöglicht. 
	\item \textbf{Keras}:\\ Open-Source-API für Deep Learning, seit TensorFlow 2.0 integraler Bestandteil der TensorFlow Core API.
	\item \textbf{TensorFlow}:\\  Open-Source-Framework für maschinelles Lernen und tiefe neuronale Netzwerke, bekannt für seine Skalierbarkeit und umfangreiche Plattformunterstützung.
\end{itemize} 

\addcontentsline{toc}{section}{Arbeitsverteilung}
\chapter*{Arbeitsverteilung}

\textbf{Teilnehmer 1:  Elisa Du} \\
\\
 Inhalte: \\
\\ Grundlagen - Kapitel (exklusiv Pix2Pix)
\\ Problembeschreibung - Kapitel
\\ Lösungsbeschreibung - Kapitel (exklusiv Implementierung der Pix2PixGAN-Architektur)
\\
\\
\\
\textbf{Teilnehmer 2: Marcel Hoffmann} \\
\\
 Inhalte: \\
\\ Pix2Pix in Grundlagen - Kapitel
\\ Implementierung der Pix2PixGAN-Architektur in Lösungsbeschreibung - Kapitel
\\
\\
\\
