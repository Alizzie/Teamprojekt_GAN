% In dieser Datei werden Aussehen und Inhalt des "Abstract" definiert.

%%% Abstract auf Englisch
\begin{addmargin}[0cm]{0cm}
		\begin{center}
			\LARGE{\textbf{Abstract}}\\
			\rule{\textwidth}{0.04cm}	
		\end{center}

Groundbreaking advances in image processing have been achieved through the continuous development of Generative Adversarial Networks (GANs). This thesis deals with the analysis, design and implementation of different GAN architectures. The focus is on the fundamentals of GANs, Pix2Pix and CycleGAN. The emphasis is on the detailed description of the generators and discriminators for Pix2Pix and CycleGAN. This is followed by an in-depth examination of the implementation and evaluation using various evaluation criteria, including generator and discriminator loss and the Structural Similarity Index (SSIM). Finally, a comparison is made between Pix2Pix and CycleGAN to show the advantages and disadvantages of each. The aim of this thesis is to provide a comprehensive overview of the diversity of GAN architectures and identify potential challenges by analysing two representative models.

% Trennung zwischen dem englischen und dem deutschen Abstract.
\vspace{1cm}  

%%% Abstract auf Deutsch
\begin{center}
			\LARGE{\textbf{Zusammenfassung}}\\
			\rule{\textwidth}{0.04cm}	
		\end{center}

Bahnbrechende Fortschritte in der Bildverarbeitung wurden durch die kontinuierliche Entwicklung von Generative Adversarial Networks (GANs) erzielt. Diese Arbeit beschäftigt sich mit der Analyse, dem Entwurf und der Implementierung verschiedener GAN-Architekturen. Der Fokus liegt dabei auf den Grundlagen von GANs, Pix2Pix und CycleGAN. Der Schwerpunkt liegt auf der detaillierten Beschreibung der Generatoren und Diskriminatoren für Pix2Pix und CycleGAN. Es folgt eine eingehende Untersuchung der Implementierung und Evaluierung anhand verschiedener Evaluierungskriterien, einschließlich Generator- und Diskriminatorverlust und dem Structural Similarity Index (SSIM). Abschließend wird ein Vergleich zwischen Pix2Pix und CycleGAN durchgeführt, um die jeweiligen Vor- und Nachteile aufzuzeigen. Das Ziel dieser Arbeit ist es, durch die Analyse von zwei repräsentativen Modellen einen umfassenden Überblick über die Vielfalt von GAN-Architekturen zu geben und mögliche Herausforderungen zu identifizieren.

\end{addmargin}
\cleardoublepage